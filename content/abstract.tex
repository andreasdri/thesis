
\begin{otherlanguage}{english} 
\begin{abstract}
Chronic diseases are the most common cause of death worldwide. By following up
on patients in their own homes by self-reporting their daily form and reporting
sensor data, the goals are to reduce hospitalizations and increase the sense of
security and achievement. Remote monitoring of patients with chronic
diseases (remote monitoring) is a priority area for
Norway, which has allocated 30 million NOK to a project to test this in
four municipalities: Sarpsborg, Oslo, Stavanger and Trondheim.

Cloud-based Internet of Things (IoT) is presented as a technology that
will solve problems with collecting data from sensors in a large scale.
New platforms such as AWS IoT has been released to support this development
with infrastructure and components.
In this master's thesis, a prototype of a stand-alone
pulse oximeter connected to AWS IoT was evaluated for use in remote monitoring.
The purpose of this was to find an architecture for remote monitoring based on modern
IoT solutions in the cloud released during the last
couple of years, and what the important aspects will be with the development of
such a technological platform in the future.
The solution was based on a Raspberry Pi Zero W, a fingerprint sensor for
authentication of the patient, one button and three LEDs.

A case study was conducted to see how the municipality of Trondheim is working
with remote monitoring, and what the current situation is in that project in the
beginning of 2017. One background interview with the municipality of Trondheim was conducted.
In the end of the project, the prototype was showed to the municipality of Trondheim
and SINTEF in two separate interviews.

The results showed that a possible architecture for remote monitoring can be
realized with the cloud platform AWS IoT, which gives good built-in safety.
The project identified four implications for the development of cloud-based
IoT as a technological platform for remote monitoring: security and authentication,
protection of privacy, developer experience and price/administration.
In regards to security and authentication, it is important to use end-to-end encryption
and two-factor authentication. Use Bluetooth version 4.2 or newer if you can.
For personal privacy, make sure that user maintains control over their own data
and think about how the new regulations (GDPR) from 2018 affect systems.
AWS IoT was simple to use from a developer's point of view, and this
saved time to work on the prototype itself. The municipality had to use a lot
of resources to own the entire value chain. It could be simpler and cheaper to
only give out sensors.

The municipality of Trondheim pointed out that the difficulty in remote
monitoring is not necessarily the technology, but making the solution work in
large scale in a municipality with hundreds of users. They would perhaps rather have a
solution with sensors around a central hub (which could have been stand-alone)
than having everything be integrated in the sensor. They projected a future
where the user can choose the solution they want to use and that the
municipality can be present on the platforms the users already are on.
The researchers from SINTEF thought that self-reporting the daily form should
have been in the prototype. Preferably, they wanted the self-treatment plan in the
solution, and not only on a piece of paper. They were optimistic about the technology when
it came to reporting of sensor data, and they viewed automatic analysis of data and
better two-way communication as something positive to work on in the
future.
\end{abstract}
\end{otherlanguage}
