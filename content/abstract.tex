
\begin{otherlanguage}{english} 
\begin{abstract}
Chronic diseases are the most common cause of death worldwide. By following up
on patients in their own homes by self-reporting their daily form and reporting
sensor data, the goals are to reduce hospitalizations and increase the sense of
security and achievement. Remote monitoring of patients with chronic
diseases (remote monitoring) is a priority area for
Norway, which has allocated 30 million NOK to a project to test this in
four municipalities: Sarpsborg, Oslo, Stavanger and Trondheim.

In this master's thesis, a prototype of a stand-alone and cloud-connected
pulse oximeter was evaluated for use in remote monitoring.
The solution was based on a Raspberry Pi Zero W, a fingerprint sensor for
authentication of the patient, one button and three LEDs. The purpose
of this was to find an architecture for remote monitoring based on modern
Internet of Things (IoT) solutions in the cloud released during the last
couple of years, and what the important aspects will be with the development of
such a technological platform in the future.

A case study was conducted to see how the municipality of Trondheim is working
with remote monitoring, and what the current situation is in that project in the
beginning of 2017. To rounds of interviews with a project manager and a person
who works with welfare technology in the municipality of Trondheim were
conducted: first a background interview and then an evaluation and
summary interview where they looked at and tested the prototype. An evaluation
and summary interview with two researchers from SINTEF was also conducted.

The results showed that a possible architecture for remote monitoring can be
realized with the cloud platform AWS IoT, which gives good built-in safety.
The problem with this platform is that the data is not stored in Norway which
has usually been the norm in Norway for health applications.
An other important technical finding, was that sensors must use new versions
of Bluetooth as early as possible, and preferably version 4.2 or later.
The new General Data Protection Regulation (GDPR) from the European Union which
will be implemented in Norway in the beginning of 2018, will give users more
control of their own data, and will have consequences for the development of
cloud IoT solutions in welfare technology. The municipality of Trondheim were
aware of the problems from a legal point of view, and said that they would
continue to work with this.

The municipality of Trondheim pointed out that the difficulty in remote
monitoring is not necessarily the technology, but making the solution work in
large scale in a municipality with hundreds of users. They thought the prototype
was easy to use and could be used by patients. They would perhaps rather have a
solution with sensors around a central hub (which could have been stand-alone)
than having everything be integrated in the sensor. They projected a future
where the user can choose the solution they want to use and that the
municipality can be present on the platforms the users already are on -- for
instance iOS, Android, PC or a stand-alone solution.

The researchers from SINTEF thought that self-reporting the daily form should
have been in the prototype. Also, a fingerprint sensor would not be a good
solution for all types of users. They were optimistic about the technology when
it came to reporting of sensor data, and they viewed automatic analysis of data and
better two-way communication as something positive to work on in the
future.
\end{abstract}
\end{otherlanguage}
