
\begin{abstract}
\noindent
Kroniske sykdommer er den mest vanlige dødsårsaken på verdensbasis.
Ved å følge opp pasienter i sitt eget hjem med egenrapportering av dagsform
og innrapportering av sensordata, ønsker man å redusere sykehusinnleggelser
og øke trygghet- og mestringsfølelsen. Avstandsoppfølging av kronisk syke
(avstandsoppfølging) er et satsingsområde for
Norge som har bevilget 30 millioner til et prosjekt for å teste ut dette i fire
kommuner: Sarpsborg, Oslo, Stavanger og Trondheim.

\noindent \newline
I denne masteroppgaven ble en frittstående prototype av et skytilkoblet pulsoksimeter
evaluert for bruk i avstandsoppfølging. Løsningen var basert på en Raspberry Pi
Zero W, en fingeravtrykksensor for autentisering av pasienten, en knapp og tre lysdioder.
Hensiktet med dette var å finne en arkitektur for avstandsoppfølging basert på moderne
tingenes internett (IoT)-skyløsninger lansert i løpet av de siste par årene, og
hva som vil bli viktige aspekter ved utviklingen av et en slik teknologisk plattform
fremover.

\noindent \newline
Det ble gjennomført en case-studie for å se hvordan Trondheim kommune jobber med
avstandsoppfølging, og hva som er status på det prosjektet i første halvdel av 2017.
To runder med intervjuer av en prosjektleder og en som jobber med velferdsteknologi
i Trondheim kommune ble gjennomført:
først et rent bakgrunnsintervju og deretter et evaluerings- og oppsummeringsintervju
hvor de fikk se på og prøve ut prototypen. Det ble også gjort et evaluerings- og
oppsummeringsintervju med to forskere fra SINTEF.

\noindent \newline
Resultatene viste at en mulig arkitektur for avstandsoppfølging kan realiseres
med skyplattformen AWS IoT som gir god innebygd sikkerhet. Problemet med denne
plattformen er at dataen ikke lagres i Norge, som som har vært et ønske
i Norge for helseapplikasjoner.
Et annet viktig teknisk aspekt som ble avdekket, var at sensorer tidlig
må ta i bruk nye Bluetooth-versjoner -- og helst versjon 4.2 eller nyere.
Den nye personsvernforordningen til EU som implementeres i norsk lov i starten
av 2018 gir brukerne enda mer kontroll over egne data, og kommer
til å få konsekvenser for utviklingen av skybaserte IoT-løsninger i velferdsteknologi.
Trondheim kommune var veldig klar over den juridiske problematikken, og ville jobbe
videre med dette.

\noindent \newline
Trondheim kommune påpekte at det vanskelige med avstandsoppfølging ikke nødvendigvis
er teknologien, men å få løsningen til å fungere i storskala i en kommune med flere
hundre brukere.
De mente at prototypen var enkelt å bruke og kunne vært brukt av
pasienter. De ville kanskje heller ha en løsning med flere sensorer rundt en sentral hub
(som godt kunne være frittstående), enn å ha alt integrert i sensoren. De
så for seg en fremtid der brukeren selv kan velge hva slags løsning de vil bruke
og at kommunen kan være til stede på de plattformene brukeren er fra før av -- det
seg være iOS, Android, PC eller en frittstående løsning.

\noindent \newline
Forskere ved SINTEF mente at
å rapportere dagsform burde vært med i prototypen. Fingeravtrykksensor ville heller
ikke vært en god løsning for alle typer brukere.
De var teknologioptimisiske når
det gjaldt innrapportering av sensordata, og så på automatisk analyse av data
og bedre toveiskommunikasjon som noe positivt å jobbe med videre. Brukerne må få
tilbakemelding og oversikt over målingene sine.
\end{abstract}
