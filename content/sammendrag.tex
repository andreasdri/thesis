
\begin{abstract}
\noindent
Kroniske sykdommer er den mest vanlige dødsårsaken på verdensbasis.
Ved å følge opp pasienter i sitt eget hjem med egenrapportering av dagsform
og innrapportering av sensordata, ønsker man å redusere sykehusinnleggelser
og øke trygghets- og mestringsfølelsen. Avstandsoppfølging av kronisk syke
(avstandsoppfølging) er et satsingsområde for
Norge som har bevilget 30 millioner til et prosjekt for å teste ut dette i fire
kommuner: Sarpsborg, Oslo, Stavanger og Trondheim.

\noindent \newline
Skybasert tingenes internett (IoT) blir presentert som en teknologi som skal løse
en mengde problemer knyttet til innsamling av sensordata i stor skala. Nye plattformer
som AWS IoT har blitt lansert for å støtte denne utviklingen med ferdig infrastruktur
og komponenter. I denne masteroppgaven ble en frittstående prototype av et pulsoksymeter
koblet til AWS IoT, evaluert for bruk i avstandsoppfølging. Hensikten med dette var
å finne en arkitektur for avstandsoppfølging basert på moderne IoT-skyløsninger, og
finne ut hva som vil bli viktige aspekter ved utviklingen av en slik teknologisk
plattform fremover. Prototypen var basert på en Raspberry Pi Zero W, en fingeravtrykkssensor
for autentisering av pasienten, en knapp og tre lysdioder.

\noindent \newline
Det ble gjennomført en case-studie for å se hvordan Trondheim kommune jobber med
avstandsoppfølging, og hva som er status på det prosjektet i første halvdel av 2017.
Et bakgrunnsintervju med Trondheim kommune ble gjennomført. På slutten av prosjektet
ble prototypen vist fram til kommunen og SINTEF i to ulike intervjuer.

\noindent \newline
Resultatene viste at en mulig arkitektur for avstandsoppfølging kan realiseres
med skyplattformen AWS IoT som gir god innebygd sikkerhet. Prosjektet identifiserte
fire implikasjoner for utviklingen av skybasert IoT som teknologisk plattform for
avstandsoppfølging: sikkerhet og autentisering, personvern, utviklingsomgivelser
og pris og administrasjon. Med tanke på sikkerhet og autentisering er det viktig
å benytte ende-til-ende-kryptering og
tofaktorautentisering. Benytt helst Bluetooth-versjon 4.2 eller nyere.
For personvern gjelder det å sørge for at brukeren kan kontrollere egne data og tenke
på hvordan de nye personvernreglene i 2018 vil påvirke systemer. AWS IoT var enkelt
å benytte for utviklere og gjorde at mer tid kunne brukes til å jobbe med selve
prototypen. Kommunen måtte bruke mye ressurser på å eie hele verdikjeden selv. Det kan
være billigere og enklere kun å dele ut sensorer.

\noindent \newline
Trondheim kommune påpekte at det vanskelige med avstandsoppfølging ikke nødvendigvis
er teknologien, men å få løsningen til å fungere i storskala i en kommune med flere
hundre brukere.
De ville kanskje heller ha en løsning med flere sensorer rundt en sentral hub
(som godt kunne være frittstående), enn å ha alt integrert i sensoren. De
så for seg en fremtid der brukeren selv skal kunne velge hva slags løsning de vil bruke
og at kommunen kan være til stede på de plattformene brukeren er fra før av.
Forskere ved SINTEF mente at rapportering av dagsform burde vært med i prototypen. Helst
ville de også ha inn egenbehandlingsplanen til brukeren inn slik at den ikke bare er på papir.
De var teknologioptimistiske når det gjaldt bruken av sensordata i fremtiden og så på
automatisk analyse av data og bedre toveiskommunikasjon som noe positivt å jobbe med videre.
\end{abstract}
