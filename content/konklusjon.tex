\chapter{Konklusjon}
\label{ch:conclusion}

\section{Forskningsspørmål 1}
\textbf{Hvordan gjøres avstandsoppfølging av kronisk syke i dag, og hvilke planer finnes for ny teknologi på dette området?}

I Trondheim kommune har over 70 brukere fått avstandsoppfølging hjemme. Det
foregår med et Android-nettbrett brukerne får utlevert. Dette nettbrettet har en
applikasjon der brukerne kan svare på hvordan dagsformen er. Spørsmålene er
tilpasset diagnosen til brukeren. I samråd med fastlegen har brukeren en
egenbehandlingsplan som iverksettes på bakgrunn av helsetilstanden. Fastlegen
har ikke direkte tilgang til avstandsoppfølgingssystemet. Koordinasjon skjer via med
helsevakta i Trondheim kommune. Noen få brukere har fått sensorer de kan bruke
til målinger. Dette gjøres også via applikasjonen. Helsevakta får tilgang til
spotmålingen og svarene på spørsmålene i en adminapplikasjon. Tilbakemeldingene
på prosjektet er positive.


\section{Forskningsspørmål 2}
\textbf{Hva er en mulig teknologistakk for realisering av skybasert \gls{iot} for avstandsoppfølging av kronisk syke,
    [og hvordan skiller dette/denne seg fra eksisterende og planlagte løsninger]?}

En mulig teknologistakk for realisering av skybasert \gls{iot} for
avstandsoppfølging er skissert i figur TODO. Denne er basert på arbeidet som er
gjort i dette prosjektet med noen små modifikasjoner. Stakken har en hub som
alle sensorene kobler seg på. Sensorene kommuniserer med huben med Bluetooth 4.2
eller nyere. Alle sensorer skal bondes med huben slik at sensorene ikke må pares
på nytt hver gang det skal sendes data. En mulig strategi for autentisering er å
bruke et klientsidesertifikat, og at dette sertifikatet er unikt for hver bruker
av løsningen. Dette er mest egnet for en backendløsning som håndterer
sertifikater, for eksempel AWS, der MQTT over \gls{tls} med egne sertifikater
støttes. Det er helt klart mulig å benytte andre metoder for autentisering slik
at brukeren slipper å skrive inn passord hele tiden. Transport av data skal
uansett alltid skje med ende-til-ende-kryptering (\gls{tls}).
% https://hanszandbelt.wordpress.com/2016/05/18/client-certificates-and-rest-apis/
% http://docs.aws.amazon.com/iot/latest/developerguide/iot-security-identity.html
% http://www.seedbox.com/en/blog/2015/06/05/oauth-2-vs-json-web-tokens-comment-securiser-un-api/

Teknologistakken har en regelbasert backend der dataen lagres fortløpende og
prosesseres i sanntid.
% @Todo: klientside, node? noe mer?


\section{Forskningsspørmål 3}
\textbf{Hvordan vurderer domeneeksperter innen velferdsteknologi frittstående skybaserte \gls{iot}-løsninger
for avstandsoppfølging av kronisk syke?}

Prosjektledere i Trondheim kommune mente at prototypen hadde tilstrekkelig
brukskvalitet til å rulles ut til brukere. De mente imidlertid at det var bedre at
sensor og hub var adskilt og ikke koblet sammen til ett smartprodukt.

Forskere ved SINTEF var enige i at sensor og hub heller kunne vært adskilt, og trakk
fram at prototypen manglet muligheten til å svare på hvordan
dagsformen er. De så for seg at man kunne utvide prototypen med flere moduler,
for eksempel en modul med muligheten til å svare på spørsmål.

\section{Forskningsspørmål 4}
\textbf{Basert på funnene fra FS1, FS2 og FS3, hva er implikasjonene for utviklingen av skybasert \gls{iot} som teknologisk plattform
    for avstandsoppfølging av kronisk syke?}

Det er noen juridiske bøyger for å benytte eksisterende skyløsninger som teknologiplattform
for avstandsoppfølging -- norsk helsedata må lagres i Norge. Dette utelukker i
praksis AWS og Azure per dags dato.

% @TODO: Fra Dag: Relatere til krav (ved siden av sikkerhet-overskriften

\subsection{Sikkerhet} % @TODO DAG: og autentisering?
Dette er implikasjonene for sikkerhet i avstandsoppfølging:

\begin{itemize}
  \item Benytt ende-til-ende-kryptering (TLS).
  \item Autentiser brukeren og benytt tofaktorautentisering.
  \item Bruk den nyeste utgaven av Bluetooth-standarden som sensoren din
  støtter. Helst utgave 4.2 eller nyere. Hvis du må bruke en eldre versjon må
  du være observant på hvilke angrepspunkter som finnes og minimere risikoen for
  sikkerhetsbrudd. Les nøye hva beste praksis er.
\end{itemize}

\subsection{Personvern}

\begin{itemize}
  \item Helsedata må lagres i Norge. Dette utelukker per dags dato skyløsninger
  som har mange nødvendige sikkerhetsprimitiver innebygd som minstestandard.
  \item Med EUs nye personvernsforordning får brukeren enda mer rett over sin egen data.
  Brukeren må kunne se hva som er lagret, slette det og tilbyderen må kunne argumentere
  med hvorfor dataen må lagres.
\end{itemize}

Mer forskning trengs på hva man kan gjøre med dataen etter at den forlater sensoren.
Avstandsoppfølging har et stort potensiale til å gjøre sanntidsberegninger på data
som kommer inn. Disse beregningene kan påvirkes av menneskelige vurderinger knyttet
opp mot hver enkelt bruker.

%@ TODO: Dag: Brukervennlighet, funksjonelle krav, pris etc.
