\chapter{Konklusjon}
\label{ch:conclusion}

\section{Forskningsspørmål 1}
\textbf{\fs{1}}

I Trondheim kommune har ca. 70 brukere fått avstandsoppfølging hjemme. Det
foregår med et Android-nettbrett brukerne får utlevert. Dette nettbrettet har en
applikasjon der brukerne kan svare på hvordan dagsformen er. Spørsmålene er
tilpasset diagnosen til brukeren. I samråd med fastlegen har brukeren en
egenbehandlingsplan på papir som iverksettes på bakgrunn av helsetilstanden. Fastlegen
har ikke direkte tilgang til avstandsoppfølgingssystemet per i dag. Koordinasjon skjer via
helsevakta i Trondheim kommune. Seks brukere har fått sensorer (pulsoksymeter, vekt, blodtrykksmåler) de kan bruke
til målinger. Dette gjøres også via applikasjonen. Helsevakta får tilgang på
spotmålingen og svarene på spørsmålene i en adminapplikasjon.
Tilbakemeldingene på prosjektet i seg selv er veldig positive.

Prosjektledelsen har hatt utfordringer med stabiliteten til sensorløsningen.
De ser også utfordringer med personvern og å la brukeren få muligheten til å administrere og se mer av sin egen data.
Kommunen jobber med å få til sertifikater på nettbrettene, men det er ikke så lett. En annen erfaring
kommunen har gjort seg, er at brukerne er veldig forskjellige. Kommunen ønsker å være tidlig ute med
ny teknologi, men det viser seg å ikke være så enkelt å få til i praksis.

\section{Forskningsspørmål 2}
\textbf{\fs{2}}

En mulig arkitektur for realisering av skybasert \gls{iot} for
avstandsoppfølging er en klient-server-arkitektur basert på arbeidet som er
gjort i dette prosjektet med noen små modifikasjoner. Arkitekturen har en hub som
alle sensorene kobler seg på. Sensorene kommuniserer med huben med Bluetooth 4.2
eller nyere. Alle sensorer skal bondes med huben slik at sensorene ikke må pares
på nytt hver gang det skal sendes data. En mulig strategi for autentisering er å
bruke et klientsidesertifikat, og at dette sertifikatet er unikt for hver bruker
av løsningen. Dette er mest egnet for en backendløsning som håndterer
sertifikater, for eksempel AWS IoT, der MQTT over \gls{tls} med egne sertifikater
støttes. Det er helt klart mulig å benytte andre metoder for autentisering slik
at brukeren slipper å skrive inn passord hele tiden. Transport av data skal
uansett alltid skje med ende-til-ende-kryptering (\gls{tls}).

Arkitekturen har en regelbasert backend der dataen lagres fortløpende og
prosesseres i sanntid. Denne arkitekturen løser ikke problemer knyttet til personvern
og datalagring. Det er noe som må implementeres på toppen av skyløsningen. AWS gir allikevel
mange gode verktøy ut av boksen for å få til dette. Det kan understøtte mange av de
tjenesteutfordringene som HelsaMi+ opplever i dag.

\section{Forskningsspørmål 3}
\textbf{\fs{3}}

Prosjektledere i Trondheim kommune mente at prototypen hadde tilstrekkelig
brukskvalitet til å rulles ut til brukere. De mente imidlertid at det kanskje var bedre at
sensor og hub var adskilt og ikke koblet sammen til ett produkt. Prisen på prototypen
ble vurdert som rimelig. Indikering av at en måling er ferdig på skjermen til pulsoksymeteret
var noe kommunen ikke hadde med i sin løsning.

Forskere ved SINTEF var enige i at sensor og hub heller kunne vært adskilt, og trakk
fram at prototypen manglet muligheten til å svare på hvordan
dagsformen er. De så for seg at man kunne utvide prototypen med flere moduler,
for eksempel en modul med muligheten til å svare på spørsmål.

\section{Forskningsspørmål 4}
\textbf{\fs{4}}

Konklusjonen til dette forskningsspørsmålet deles opp
i avsnittene sikkerhet og autentisering, personvern, utviklingsomgivelser og
pris.

\subsection{Sikkerhet og autentisering}
\begin{itemize}
  \item Benytt ende-til-ende-kryptering (TLS).
  \item Autentiser brukeren med sertifikater eller tokens og benytt tofaktorautentisering.
  \item En moderne skytjeneste som AWS med sin løsning AWS IoT gir
      god innebygd sikkerhet som standard med TLS og sertifikater. Dette gjelder fra klient til skyen,
      men i en del tilfeller vil man ha et sikkerhetsproblem mellom klient og sensor.
  \item Bruk den nyeste utgaven av Bluetooth-standarden mellom klient og sensor som sensoren din
  støtter. Helst utgave 4.2 eller nyere. Hvis du må bruke en eldre versjon må
  du være observant på hvilke angrepspunkter som finnes og minimere risikoen for
  sikkerhetsbrudd. Les nøye hva beste praksis er.
\end{itemize}

\subsection{Personvern}
\begin{itemize}
  \item Helsedata lagres primært i Norge i dag. Dette utelukker skyløsninger
  som har mange nødvendige sikkerhetsprimitiver innebygd som minstestandard, selv
  om personopplysningsloven åpner for å flytte data til utlandet dersom visse kriterier er oppfylt.
  I skrivende stund tilbyr verken AWS eller Microsoft skyløsninger med lagring i Norge, men dette er et rent
  praktisk problem. Det har ikke noe med teknologien å gjøre.
  \item Med EUs nye personvernsforordning får brukeren enda mer rett over sin egen data.
  Brukeren må kunne se hva som er lagret, slette det og tilbyderen må kunne argumentere
  med hvorfor dataen må lagres. Dette kompliserer utviklingen av tjenester til bruk i avstandsoppfølging
  både juridisk og teknisk. Gode løsninger for brukere og administratorer må utvikles for å håndtere dette
  på en god måte. En moderne skytjeneste kan avlaste disse utfordringene ved å gi utviklere og administratorer
  gode verktøy og et helhetlig økosystem for integrasjon slik at implementasjonen blir enklere.
\end{itemize}

\subsection{Utviklingsomgivelser}
\begin{itemize}
    \item AWS IoT var enkelt å komme i gang med for en utvikler og har mange gode
        verktøy som egner seg for avstandsoppfølging av kronisk syke.
    \item Plattformen har flere ulike grensesnitt for tilkobling og integreringen
        mot andre tjenester i økosystemet er veldig god.
    \item Kommunen var imponert over
        at prototypen fungerte og hvor kort tid det hadde tatt å utvikle den.
        Skyplattformen gjorde at det var mulig å komme raskt i gang slik at det
        var mulig å bruke mer tid på prototypen på klientsiden.
\end{itemize}

\iffalse
\item Brukere og helsepersonell i avstandsoppfølging fortjener gode og bra utformet løsninger. Det er fortsatt en jobb
        å gjøre på det området. Helsepersonell savner å kunne se grafer av måledata over tid, og kan derfor ikke få så god
        nytte av dataen per i dag. Her må leverandørene jobbe tett sammen med interesseholderne for å møte behovene.
\item Leverandøren av tekniske løsninger må sette seg inn i teknologien som brukes for å gi brukeren en best mulig opplevelse.
        Det gjelder for eksempel å indikere at en måling er ferdig på skjermen til sensoren.
    \item Brukerne sier, i følge Trondheim kommune, at de synes HelsaMi+ er enkelt å bruke.
\fi

\subsection{Pris og administrasjon av utstyr}
\begin{itemize}
    \item Det er en stor kostnad for kommunen å administrere og vedlikeholde nettbrett.
    \item Direktoratet for e-helse anbefaler at kommunen skal eie hele verdikjeden, og kommunen eier
        dermed SIM-kort og abonnement for tilkobling til nettet. Det er potensiale for å se på
        effektivisering her dersom brukeren kan ha utstyret selv, eller bruke 3G som ekstrametode
        for tilkobling hvis det ikke går an å koble på WIFI.
    \item Det kan være billigere for kommunen å være til stede på de plattformene brukeren er fra før av.
        Det kan føre til lavere administrasjonskostnader. Kommunen trenger i så fall kun å dele ut sensorer.
\end{itemize}

\section{Refleksjoner rundt overordnet forskningsspørsmål}
\textbf{Hvor egnet er skybasert IoT som teknologisk plattform for avstandsoppfølging av kronisk syke?}

Som utvikler var det enkelt å jobbe med AWS IoT som teknologisk plattform for avstandsoppfølging av kronisk
syke. Denne plattformen bygger på det som har blitt en ganske standard arkitektur for \gls{iot}
med innebygget infrastruktur for å håndtere ting og hvordan tingene kobles til skyløsningen på en enkel og sikker
måte. Prosjektet hadde stor tillit til plattformen og det var behagelig at sikkerhet var standard
og med fra starten av. Det førte til at prosjektet kunne konsentere seg om å jobbe med prototypen som tok
mesteparten av utviklingstiden. Det tok svært kort tid å sette opp AWS IoT.

AWS IoT er veldig godt dokumentert og kan konfigureres og brukes med nettside, kommandolinjeverktøy og
flere forskjellige programmeringsspråk. For utviklere er det behagelig å ha mange verktøy i verktøyskassen,
uten at man må lage alle verktøy fra bunnen av selv.

AWS IoT er en proprietær og kommersiell
tjeneste som baserer seg på ganske standard arkitektur. Ulempen med å velge en slik leverandør er at man kan
bli ganske låst til leverandøren og økosystemet til leverandøren. Det er ikke helt klart om det er enkelt
å bytte mellom ulike leverandører. Kanskje er det mulig å skrive abstraksjoner som gjør at den underliggende
plattformen kan byttes ut. Fordelen er at man aldri trenger å tenke på skaleringen til tjenesten
og fysisk driftssikkerhet. Det er en styrke for plattformen at selv om den er proprietær, så baserer den seg på åpne protokoller
(MQTT) der det er mulig.
AWS IoT vil være svært godt egnet dersom kommuner vil samle inn enda mer
data i fremtiden, for eksempel hvis en bruker går med en sensor over lengre perioder hver dag.


\section{Videre arbeid}
Det er mye som kan jobbes med videre innenfor avstandsoppfølging. Spesielt lovende virker det å få til
en høyere grad av toveiskommunikasjon. Brukeren kan for eksempel få et varsel når det er på tide å gjøre en måling.
Det er mulig å bake inn kommunikasjon med Helsevakta inne i applikasjonen, enten med chat eller med videosamtale.
Sanntidsanalyse av data basert på grenseverdier hadde vært teknologisk mulig å få til uten for mye styr. Helsevakta
kunne fått opp et varsel i sin adminapplikasjon. Om brukeren selv skulle fått opp et slikt varsel måtte i så fall blitt
individuelt tilpasset.
Som erfaringene til Trondheim kommune viser har ikke alle så godt av å vite for mye om sin egen
helsetilstand. SINTEF sa at det kunne være mye positivt i at en bruker blir kjent med sine egne målinger.
Analyse av data tilbake i tid kunne også vært gjort. Her peker fastlegen på at alt ikke er så klart og
at det trengs mer forskning og flere erfaringer.

Egenbehandlingsplanen til brukeren må bakes inn i alle løsninger slik at brukeren får et bedre forhold til den.
Basert på hva brukeren rapporterer inn av dagsform og sensordata må det komme opp hvilke tiltak brukeren skal
iverksette. Dette vil øke egenmestringen til brukeren.

Det er også potensiale i overvåke alle sensorene på en bedre måte, slik at en kan følge med på hvor mye batteri som er igjen
og hva kvaliteten på sensordataen er. Her må en benytte alle de mulighetene sensoren gir for integrasjon, som for eksempel
checkmark når brukeren er ferdig med en måling. Teknologi som device shadow og device twins (delkapittel \ref{sec:deviceshadows}
og \ref{sec:azureiot}) kan brukes til å holde styr på tilstanden og manipulere tilstanden fra ulike applikasjoner.
Kommunen ønsker å være tidlig ute med
ny teknologi, men det viser seg å ikke være så enkelt å få til i praksis.

Mye kan gjøres i utviklingen av applikasjoner på eksisterende plattformer (Android, iOS, web). Disse applikasjonene bør helst være native for best
mulig ytelse og brukeropplevelse. Her kan en med fordel nyttiggjøre seg av ny teknologi og nye løsninger laget fra bunnen av for å få til dette på en bedre måte.
