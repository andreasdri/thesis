\chapter{Introduksjon}
\label{ch:introduction}
Hensikten med dette forskningsprosjektet er å utvikle ny teknologi for avstandsoppfølging av kronisk
syke i form av en første prototype av et smart pulsoksimeter som alltid er på nett. Prototypen
vil vises fram for prosjektledere og forskere innen velferdsteknologi for
å avdekke hva som vil bli viktige aspekter ved utviklingen av helseprodukter
basert på \gls{iot}-skyløsninger i fremtiden. Velferdsteknologiprogrammet i Trondheim kommune og
SINTEF Avdeling Helse bistår prosjektet. 

Dette kapittelet beskriver bakgrunnen for forskningsprosjektet: hva som motiverer det,
forskningsspørsmålene som skal besvares og forskingsdesignet som understøtter det,
hvilke begrensninger og avveininger som er gjort og til slutt en disposisjon av oppgaven.

\section{Bakgrunn og motivasjon}
\gls{iot} gir nye og spennende muligheter for kreative løsninger på problemer på forskjellige områder, alt fra husholdningsapparater
og fjernoppdatering av programvare i biler, til oppfølging av helseproblemer innen velferdsteknologi. Datakraft- og lagring
har blitt mye billigere, noe som gjør at man i større grad enn tidligere kan samle inn informasjon om verden rundt oss
med sensorer. De to største leverandørene av skytjenester, \gls{aws} og Microsoft Azure,
har i løpet av det siste halvannet året lansert nye tjenester for å få sensorenheter tilkoblet til Internett
på en enkel og sikker måte.

Velferdsteknologi har vært høyt prioritert av helsemyndighetene i Norge de siste årene. Trondheim kommune prøver ut
et prosjekt med avstandsoppfølging av kronisk syke kalt HelsaMi+ på oppdrag fra Helsedirektoratet. 
Avstandsoppfølging kan øke trygghetsfølelsen for innbyggerne og føre til lavere kostnader og færre sykehusinnleggelser (kilde?).
Løsningen til Trondheim kommune innebærer bruk av et nettbrett der man daglig svarer på spørsmål om hvordan formen er. Noen av pasientene
kan måle vekt, blodtrykk, pulsfrekvens og oksygenmetning ved hjelp av sensorer koblet til nettbrettet.

Motivasjonen for dette forskningsprosjektet er å ta i bruk den nye skyteknologien fra \gls{aws} i en frittstående
pulsoksimeterprototype for avstandsoppfølging av kronisk syke, og lage noe annerledes enn den løsningen Trondheim
kommune tester ut i dag. Det vil gjøre at dette prosjektet kan sammenligne de to løsningene og undersøke hvordan ny
teknologi kan brukes i avstandsoppfølging av kronisk syke.

\section{Forskningsspørsmål}
\label{sec:res_questions}
Det overordnete forskningsspørsmålet er: «Hvor egnet er skybasert \acrfull{iot} som teknologisk plattform for avstandsoppfølging av kronisk syke?»
Dette spørsmålet legger premissene for de konkrete forskningsspørsmålene under:

\begin{enumerate}
  \item[\textbf{FS1}] Hvordan gjøres avstandsoppfølging av kronisk syke i dag, og hvilke planer finnes for ny teknologi på dette området?
  \item[\textbf{FS2}] Hva er en mulig teknologistakk for realisering av skybasert \gls{iot} for avstandsoppfølging av kronisk syke,
    [og hvordan skiller dette/denne seg fra eksisterende og planlagte løsninger]?
  \item[\textbf{FS3}] Hvordan vurderer domeneeksperter innen velferdsteknologi frittstående skybaserte \gls{iot}-løsninger for avstandsoppfølging av kronisk syke?
  \item[\textbf{FS4}] Basert på funnene fra FS1, FS2 og FS3, hva er implikasjonene for utviklingen av skybasert \gls{iot} som teknologisk plattform
    for avstandsoppfølging av kronisk syke?
\end{enumerate}

Heretter vil «avstandsoppfølging av kronisk syke» bli betegnet som «avstandsoppfølging».

\section{Forskningsmetoder og forskningsdesign}
Forskningsprosjektet baserer seg på prosessen beskrevet i \citet{oates}. Hovedstrategien for å besvare på \textbf{FS1} i
seksjon \ref{sec:res_questions} er å gjøre en liten eksempelstudie på hvordan avstandsoppfølging foregår i Trondheim kommune. Datagenereringsmetoder
er intervjuer og dokumenter. 

For å svare på \textbf{FS2} og \textbf{FS3} vil den primære strategien være «design og kreasjon» (prototyping/«proof of concept»)
med intervjuer og dokumenter som datageneringsmetode. Eksempelstudiet vil også hjelpe til med å besvare disse spørsmålene. \textbf{FS4} vil drøftes kvalitativt 
utifra svarene på de andre forskningsspørsmålene. Det vil kun gjøres kvalitative dataanalyser. De to forskningsstiene er vist i figur ref1 og figur ref2.

Forskningsmetodene og forskningsdesignet er utdypet i kapittel \ref{ch:method}.

\section{Avgrensning av forskningen}
Om man kan gjøre kliniske vurderinger basert på sensordataene, kommer ikke til å være i søkelyset for dette prosjektet,
men vil nevnes kort i et senere kapittel. Det er noe som kunne vært problematisert i en annen oppgave.
Denne forskningen kommer til å anta at det er nyttig å samle inn helsedata fra sensorer for å kartlegge helsetilstanden til kroniske pasienter.

Prototypen gjennomgår ikke brukbarhetstesting. En annen tilnærming til oppgaven kunne vært å gjennomføre brukertester på pasienter eller andre brukere, men det
fører med seg krav om behandling av sensitive helseopplysninger. 

\section{Deltakere i prosjektet}
Fra Trondheim kommune sin side er Ingar Børre Sandvik (prosjektleder, HelsaMi+, program for velferdsteknologi) og Renate Enger
(delprosjektleder HelsaMi+, program for velferdsteknologi+) involvert. Kommunen låner ut et pulsoksimeter.

Anita Das (forsker ved SINTEF teknologi og helse, avdeling helse) er med på evalueringen fra SINTEF. SINTEF gjør mye forskning på
avstandsoppfølging, og bidro også til å få i gang et samarbeid med Trondheim kommune.

Terje Røsand (overingeniør, institutt for datateknologi og informatikk) bistår med teknisk hjelp til prototyping og 3D-printing.

Alle aktørene er gjort kjent med formålet for forskningsprosjektet og hva den innsamlede dataen skal brukes til. De er informert om at
intervjuene tas opp på telefon.

\section{Disposisjon av oppgaven}
Neste kapittel gir en teoretisk bakgrunn innenfor velferdsteknologifeltet og avstandsoppfølging av kronisk syke spesielt.
Det inneholde også bakgrunn om \gls{iot} og bruken av \gls{iot} i velferdsteknologi.

Kapittel \ref{ch:method} beskriver hvilke forskningsmetoder og hvilket forskningsdesign som brukes i prosjektet for å besvare forskningsspørsmålene. 
Det inneholderr også en drøfting om hvorvidt forskningsmetodene er hensiktsmessige, og hva validiteten til forskningsfunnene er.

Kapittel \ref{ch:case} handler om avstandsoppfølging i Trondheim kommune som en eksempelstudie. Det baserer seg på intervjuer gjort med
prosjektledere og skjermbilder av løsningen.

I kapittel \ref{ch:requirements}, drøftes krav til en frittstående \gls{iot}-løsning til bruk i avstandsoppfølging basert på krav fra Helsedirektoratet,
Datatilsynet og eksempelstudien.

Kapittel \ref{ch:technology} går igjennom teknologi som er relevant for implementasjonen av et frittstående og skytilkoblet pulsoksimeter,
før kapittel \ref{ch:implementation1} går i detalj på hvordan prototypen ble laget og hvordan den ble seende ut (+noe om videre arbeid?).

Evalueringen av prototypen skjer i kapittel \ref{ch:evaluation1}, der resultatene av intervjuene gjort hos SINTEF og Trondheim kommune oppsummeres.

Avslutningsvis omhandler kapittel \ref{ch:discussion} diskusjon og drøfting av resultatene fra evalueringene, og konklusjonen i kapittel \ref{ch:conclusion}
oppsummerer funnene i oppgaven med bakgrunn i forskningsspørsmålene, med en del om videre arbeid.