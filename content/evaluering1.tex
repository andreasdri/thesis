\chapter{Evaluering av prototype}
\label{ch:evaluation1}
Kapittelet beskriver resultatet av evalueringene gjort med SINTEF og Trondheim kommune.

\section{Planlegging av evaluering}
Kapittel \ref{ch:design} redegjør for og drøfter valgene av evalueringsmetoder.
Evalueringsmetoden er semistrukturerte intervjuer med innlagt demonstrasjon av prototype. Delkapittel \ref{sec:innsamling}
beskriver hvordan kontakten med Trondheim kommune ble opprettet.

Intervjuforespørselen til evaluering- og oppsummeringsintervju ligger vedlagt i tillegg \ref{appendix:invitasjon_evaluering},
og ble sendt ut til Trondheim kommune ved Renate Enger og SINTEF. Alle intervjuobjekter fikk beskjed
om at intervjuet ble tatt opp med telefon og at lydopptaket blir slettet. I etterkant ble det informert om at alle deltakerne
kunne lese igjennom transkriberingen fra intervjuet, og trekke seg fra prosjektet når som helst. De signerte anonyme og ikke-anonyme
samtykkeerklæringer (vedlegg \ref{appendix:samtykke}).

Hensikten med intervjuene var å få tilbakemeldinger
på den utviklede prototypen og ha en åpen samtale om hvilke løsninger som kan egne seg for avstandsoppfølging og hvordan utviklingen blir
på dette feltet i fremtiden. Det var ikke en brukertest, men Trondheim kommune fikk prøve løsningen selv også.
Invitasjonen resulterte i to separate gruppeintervjuer, ett intervju med Trondheim kommune representert ved
R. Enger og I. B. Sandvik og ett intervju med to forskere fra SINTEF.

\section{Intervju med Trondheim
kommune}\label{intervju-med-trondheim-kommune}
Intervjuet ble gjennomført tirsdag 2. mai 2017 kl. 09.30 til rundt kl. 10.20 i lokalene til Trygghetspatruljen i Trondheim kommune
i Klæbuveien. I. B. Sandvik måtte gå kl. 10.00. Intervjuet startet med en gjennomgang og forklaringen av prototypen, før prototypen
ble demonstrert. I. B Sandvik la inn sitt eget fingeravtrykk og fikk prøve prototypen selv. Deretter dreide samtalen inn på tilbakemeldinger,
nåværende status på HelsaMi+ og utviklingen på velferdsteknologifeltet og hvilke løsninger de så for seg i fremtiden. Intervjuet ble transkribert,
og relevante deler ble fargekodet for temaene tilbakemeldinger, brukeropplevelse, pris, sikkerhet og autentisering og fremtidige løsninger.
Alle disse temaene ble relatert til den løsningen som Trondheim kommune har i dag.

\subsection{Tilbakemeldinger på prototypen}
De generelle tilbakemeldingene var at prototypen var enkel og at de var imponert over hvor kort tid det hadde tatt å utvikle den.
\textquote[R. Enger, 2. mai, personlig kommunikasjon]{Jeg synes konseptet er veldig bra. Det virka. Jeg synes jo at den prototypen er brukervennlig}{.}
\blockquote[I. B. Sandvik 2. mai, personlig kommunikasjon]{Nei, så konseptet her er jo veldig interessant da. Og det viser jo at det er relativt enkelt å få til da.
Du har jo sikkert brukt en del timer på det da. (...) Det var jo veldig enkelt å forstå konseptet her, sant. Du brukte jo bare ett og to minutt å forklare (...)}{.}

Senere i samtalen:\newline
\textbf{R. Enger:} \textquote{Ja den der ville ha fungert hjemme hos veldig mange av våre brukere.} \newline
\textbf{I. B. Sandvik}: \blockquote{Sant. Da tenker jeg at -- la oss si da -- tenkt deg brukerscenario da at noen brukere kan være -- det er liksom SpO2-måler som er
    interessant, trenger ikke å rapportere noe annet, ikke sant. Så sier vi at du tar dette her tre ganger i uka. Og så bare tar du målinga og så er dataen sendt. That's
    it. du trenger ikke å slå på noen nettbrett -- du trenger ikke å slå på noen pc, du bare, hvis du ikke vil se grafen din som du kan, men, det er det du gjør, ikke
sant. Tar målingen. Så får du kvittert ferdig (...).}

\textbf{I. B. Sandvik:}
\textquote{Jeg må si at innenfor de rammene du har så syntes jeg det her ble bra jeg.} (...)
\textquote{Så er det litt artig å se ting som fungerer og ikke bare -- det er nok av leverandører som prater om hvor bra alt skal bli.}

Kommunen sa at Nonin kunne tenkt på et lignende produkt: \textquote[I. B. Sandvik]{Det første som slår meg det er jo dette med at sånn som Nonin som lager denne her, de
kunne jo på en måte ha laget alt sammen i den enheten da sant. Som kanskje hadde blitt litt større.}

\subsection{Brukeropplevelse}
En workshop med brukere av HelsaMi+ viste, i følge I. B. Sandvik, at de fleste brukerne syntes nåværende løsning var enkel. Dette var noe som overrasket
kommunen litt, i og med at de ikke syntes at nåværende brukerapplikasjon var så imponerende.
\textquote{Og vi hadde jo ønsket oss at leverandøren kunne ha vært litt mer framoverlent og litt mer i forkant av ting da. Men det er nå på en måte det vi har da.
Men det interessante det var at brukerne synes at det var enkelt og greit.}

Aldersspennet på brukerne som var med på workshopen var fra 58 opp til rundt 82. R. Enger sa at den yngste brukeren som har HelsaMi+ er 52 år.

At brukeren får et checkmark opp på skjermen til pulsoksymeteret når målingen er ferdig, var noe kommunen ikke hadde i deres løsning. Det var
de interessert i å få med, og de tok bilde av det. I.B Sandvik sa: \textquote{Det der burde vi ha fått fiksa på. Det på en måte -- det gir en endelig
    beskjed til brukeren at den er ferdig.}

\subsection{Pris}
I. B. Sandvik spurte om prisen på systemet. Han syntes løsningen var billig. Merk at svaret om prisen på fingeravtrykksensoren var feil.
Det ble oppgitt at den kostet 25 dollar når den egentlige prisen er 50 dollar. Prisen på Raspberry Pi Zero W er 10 dollar, men det kan
være vanskelig å få tak i kun enheten uten å kjøpe en startpakke i tillegg.

\begin{description}[leftmargin=!,labelwidth=\widthof{\bfseries Knapp, lysdioder, annet}]
    \item[Raspberry Pi Zero W:] ca. 100 kr (thepihut.com, 6. juni 2017)
    \item[GT511-C3:] ca. 420 kr (sparkfun.com, 6. juni 2017)
    \item[Knapp, lysdioder, annet:] ca. 30 kr (høyt estimat)
    \item[Totalsum:] ca. 550 kr
\end{description}

\subsection{Sikkerhet og autentisering}
Kommunen var i gang med å få på plass sertifikater på nettbrettene, men I. B. Sandvik sa at de hadde problemer med å finne en god metode
for å distribuere og håndtere sertifikatene:
\blockquote{(...) vi skal prøve å få implementert sertifikat på nettbrettene, på devicene. Så har vi en sak gående med Evry, som er kommunen sin part, leverandør.
    Men vi har
    ikke fått noen gode løsninger på hvordan vi skal gjøre det her. Hvordan skal du distribuere et sånt sertifikat ut til et nettbrett som er levert ut til en
    privatperson.

(...) for det vi holder på med og så se på sammen med e-helse og Norsk Helsenett, det er nå og så teste ut at vi kan legge
sertifikatsdistribusjonen som en komponent i
kall det -- helseapplikasjonen da. Slik at du kan pushe ut sertifikater i fra applikasjonen til devicen.}

\subsection{Fremtidige løsninger i avstandsoppfølging og HelsaMi+}
Kommunen har erfart at brukerne er veldig forskjellige. Noen er veldig komfortable med teknologi og føler seg hjemme, mens andre
kun trykker på det de har fått opplæring i.
\blockquote{
(...) man gjøre en individuell vurdering da -- okei, for deg som er aktuell for tjenesten -- okei, har du nettbrett i dag? Ja.
Da kan vi kanskje bare pushe ut en applikasjon som
du får tilgang til med sikkerheten. Har du nettbrett eller er du familiær med teknologi? Nei aldri brukt det før -- okei,
da har vi utstyr som du kan få låne av kommunen og du får opplæring, ikke sant.
}

I. B. Sandvik sin konklusjon var at kommunen må tilpasse tjenesten til brukeren. Det må gjøre en individuell vurdering i hvert enkelt tilfelle --
dette passer også bra inn i kommunens strategi ellers. Å være til stede på de plattformene brukeren er på gjør også at kommunen kan
bruke færre ressurser på å eie og vedlikeholde masse utstyr.

Brukerne var forskjellige når det gjaldt å ha tilgang på egne målinger: \textquote[I. B. Sandvik]{(...) noen tenkte jo at 'ja, sikkert ikke så dumt'.
mens andre 'nei, ikke interessert'. Det
handla også sikkert om en del andre ting også. Diagnosen og ja, vil helst ikke forholde seg til det}{.}
Det er ikke et krav at brukeren må se sine egne data om brukeren ikke har lyst: \textquote[I. B. Sandvik]{(...) det trenger ikke nødvendigvis å være
noe som du må ha tilgang til som bruker da, men at det -- du skal kunne ha tilgang til det}{.}

Kommunen fikk forespeilet en løsning hvor sensoren var utenfor huben. Det vil si at produktet ikke er fullintegrert, men består
av en løsning med flere sensorer rundt enn hub som kan ligne på den som ble utviklet i dette prosjektet. I. B Sandvik sa
at en slik løsning kanskje hadde vært bedre. Den løsningen hadde gitt mulighet til å koble på blodtrykksmåler og vekt i tillegg til
pulsoksymeter.

Kommunen ville holde seg på seks brukere med sensorer ut året fordi det var nok å administere alt med de stabilitetsproblemene som
som var i løsningen i dag.
\textquote[R. Enger]{(...) og så har vi for dårlig grafer og sånn til at vi kan gjøre oss noe god faglig nytte av det per i dag da. Så vi
    er veldig i utvkling med leverandøren. Men de som har sensorer er jo veldig fornøyd. Det er de.
De synes jo at det er en ekstra trygghet å få vurdert sensordataen sine}{.} Fastlegen skal kobles på for å sette grenseverdier på sensordataen,
men det har ikke skjedd ennå i påvente på at løsningen skal bli litt mer stabil.


% For tettere mellomrom:
%\providecommand{\tightlist}{%
%  \setlength{\itemsep}{0pt}\setlength{\parskip}{0pt}}

\section{Intervju med SINTEF}\label{intervju-med-sintef}

\begin{itemize}
\tightlist
\item
  Tilstede: Anita Das og Kristine Holbø
\item
  Dato: mandag 8. mai kl. 10.30
\item
  Sted: SINTEF Teknologi og samfunn på Lerkendal
\item
  Formål: Demonstrere og vise fram prototypen. Ha en samtale om
  avstandsoppfølging og hva som vil bli viktig i utviklingen av
  produkter for avstandsoppfølging i fremtiden.
\item
  Stil: Uformelt, ikke noe brukertest. Semistrukturert intervju der
  intervjueren hadde forberedt samtaleemner og noen spørsmål.
\item
  Innhold

  \begin{itemize}
  \tightlist
  \item
    Introduksjon av meg og prosjektet.
  \item
    Introduksjon av intervjuobjekter og hva de jobber med.
  \item
    Samtale rundt velferdsteknologi og avstandsoppfølging
  \item
    Visning av prototype
  \item
    Samtale om hva de synes om prototypen og hva som vil bli viktig for
    utviklingen av produkter for avstandsoppfølging i framtiden.
  \end{itemize}
\item
  Oppsummering og evaluering

  \begin{itemize}
  \tightlist
  \item
    Sarpsborg kommune har flere pasienter som bruker sensorer og en
    annen teknologileverandør.
  \item
    Brukerne er forskjellige. Noen har veldig stor tillit til å levere
    fra seg måledata, mens andre er veldig skeptiske til hele konseptet.
  \item
    Tilbakemeldingene på prosjektet i Trondheim kommune som helhet er
    veldig positive.
  \item
    Bluetooth: Snakket om sikkerhet i Bluetooth og viktigheten av å være
    raskt ute med ny teknologi. De var ikke klar over
    sikkerhetsproblematikken til Bluetooth 4.0 og at pulsoksymeteret
    bruker denne standarden.
  \item
    Prototype: Muligheten til å kunne melde inn dagsform (subjektiv) må
    være tilstede i løsningen. Det gir økt trygghet for brukeren.
  \item
    Prototype: Fingeravtrykksensor vil ikke fungere som
    autentiseringsløsning for alle brukere. Noen har dårlig
    blodgjennomstrømning og tykke fingre. Må finne på nye måter for å
    autentisere slik at det blir tofaktor (sertifikat + noe annet).
  \item
    Prototype: De likte også bedre at huben var for seg selv og at
    sensorene var løse.
  \item
    Fremtiden: Så for seg en løsning som flere moduler. Slik at man kan bygge
    sammen alt som flere klosser for å tilpasse det til forskjellige
    brukere. For eksempel en modul for å rapportere dagsform.
  \item
    Fremtiden: De hadde veldig stor tro på innrapportering av sensordata.
    Teknologioptimistiske. Mente det var fremtiden. Så for seg
    automatisk analyse av data og bedre toveiskommunikasjon som et
    positivt hjelpemiddel.
  \end{itemize}
\end{itemize}
