\chapter{Evaluering av prototype}
\label{ch:evaluation1}
Kapittelet beskriver resultatet av evalueringene gjort med SINTEF og Trondheim kommune.

\section{Planlegging av evaluering}
Kapittel \ref{ch:design} redegjør for og drøfter valgene av evalueringsmetoder.
Evalueringsmetoden er semistrukturerte intervjuer med innlagt demonstrasjon av prototype. Delkapittel \ref{sec:innsamling}
beskriver hvordan kontakten med Trondheim kommune ble opprettet.

Intervjuforespørselen til evaluering- og oppsummeringsintervju ligger vedlagt i tillegg \ref{appendix:invitasjon_evaluering},
og ble sendt ut til Trondheim kommune ved Renate Enger og SINTEF. Alle intervjuobjekter fikk beskjed
om at intervjuet ble tatt opp med telefon og at lydopptaket blir slettet. I etterkant ble det informert om at alle deltakerne
kunne lese igjennom transkriberingen fra intervjuet, og trekke seg fra prosjektet når som helst. De signerte anonyme og ikke-anonyme
samtykkeerklæringer (vedlegg \ref{appendix:samtykke}).

Hensikten med intervjuene var å få tilbakemeldinger
på den utviklede prototypen og ha en åpen samtale om hvilke løsninger som kan egne seg for avstandsoppfølging og hvordan utviklingen blir
på dette feltet i fremtiden. Det var ikke en brukertest, men Trondheim kommune fikk prøve løsningen selv også.
Invitasjonen resulterte i to separate gruppeintervjuer, ett intervju med Trondheim kommune representert ved
R. Enger og I. B. Sandvik og ett intervju med to forskere fra SINTEF.

\section{Intervju med Trondheim
kommune}\label{intervju-med-trondheim-kommune}
Intervjuet ble gjennomført tirsdag 2. mai 2017 kl. 09.30 til rundt kl. 10.20 i lokalene til Trygghetspatruljen i Trondheim kommune
i Klæbuveien. I. B. Sandvik måtte gå kl. 10.00. Intervjuet startet med en gjennomgang og forklaringen av prototypen, før prototypen
ble demonstrert. I. B Sandvik la inn sitt eget fingeravtrykk og fikk prøve prototypen selv. Deretter dreide samtalen inn på tilbakemeldinger,
nåværende status på HelsaMi+ og utviklingen på velferdsteknologifeltet og hvilke løsninger de så for seg i fremtiden. Intervjuet ble transkribert,
og relevante deler ble fargekodet for temaene tilbakemeldinger, brukeropplevelse, pris, sikkerhet og autentisering og fremtidige løsninger.
Alle disse temaene ble relatert til den løsningen som Trondheim kommune har i dag.

\subsection{Tilbakemeldinger på prototypen}
De generelle tilbakemeldingene var at prototypen var enkel og at de var imponert over hvor kort tid det hadde tatt å utvikle den.
\textquote[R. Enger, 2. mai, personlig kommunikasjon]{Jeg synes konseptet er veldig bra. Det virka. Jeg synes jo at den prototypen er brukervennlig}{.}

\subsection{Brukeropplevelse}
\subsection{Pris}
I. B. Sandvik spurte om prisen på systemet. Han syntes løsningen var billig. Merk at svaret om prisen på fingeravtrykksensoren var feil.
Det ble oppgitt at den kostet 25 dollar når den egentlige prisen er 50 dollar. Prisen på Raspberry Pi Zero W er 10 dollar, men det kan
være vanskelig å få tak i kun enheten uten å kjøpe en startpakke i tillegg.

\begin{description}[leftmargin=!,labelwidth=\widthof{\bfseries Knapp, lysdioder, annet}]
    \item[Raspberry Pi Zero W:] ca. 100 kr (thepihut.com, 6. juni 2017)
    \item[GT511-C3:] ca. 420 kr (sparkfun.com, 6. juni 2017)
    \item[Knapp, lysdioder, annet:] ca. 30 kr (høyt estimat)
    \item[Totalsum:] ca. 550 kr
\end{description}

\subsection{Sikkerhet og autentisering}
\subsection{Fremtidige løsninger}


% For tettere mellomrom:
%\providecommand{\tightlist}{%
%  \setlength{\itemsep}{0pt}\setlength{\parskip}{0pt}}

\def\tightlist{}

\begin{itemize}
\tightlist
\item
  Tilstede: Renate Enger (09.30-10.15) og Ingar Børre Sandvik
  (09.30-10.00)
\item
  Dato: tirsdag 2. mai kl. 09.30 til kl. 10.15
\item
  Sted: Nidarvoll helsehus
\item
  Formål: Demonstrere og vise fram prototypen. Gi muligheten til på
  prøve selv. Ha en samtale om hva som vil bli viktig i utviklingen av
  produkter for avstandsoppfølging i fremtiden.
\item
  Stil: Uformelt, ikke noe brukertest. Semistrukturert intervju etterpå
  der intervjueren hadde forberedt samtaleemner og noen spørsmål.
\item
  Oppsummering av tilbakemeldinger på prototypen

  \begin{itemize}
  \tightlist
  \item
    Veldig enkelt å bruke.
  \item
    Imponert over hvor kort tid det hadde tatt å utvikle den, og hvor
    billig løsningen var.
  \item
    Kunne vært rullet ut for pasienter.
  \item
    De var mer optimistiske på vegne av løse sensorer rundt en liten hub
    enn å integrere sensoren i selve produktet. Dette passer også bedre
    om en kommune skal bruke flere forskjellige sensorer.
  \item
    De mente at Nonin burde utviklet et pulsoksymeter som var mer
    integrert. Vi snakket litt om at det er enklere for Nonin å kun
    utvikle måleteknologien og gi instruksjoner for hvordan man henter
    ut data enn å lage en liten datamaskin.
  \item
    At det kommer opp et ``checkmark'' på pulsoksymeteret når en måling
    er ferdig var noe de ikke hadde i sin egen løsning i dag, og var
    interessert i å implementere selv.
  \item
    Trondheim kommune gjør en ``spot-check''-måling (kun én måling som
    er verifisert bra sendes) og ikke kontinuerlige målinger.
  \end{itemize}
\item
  Oppsummering av tanker om fremtiden

  \begin{itemize}
  \tightlist
  \item
    Trondheim kommune ønsker å være tilstede på de plattformene brukerne
    er fra før av, det seg være iOS, Android, PC osv.
  \item
    De ser for seg i fremtiden at brukeren selv kan velge hva slags
    løsning de vil bruke og at de slipper å levere ut et eget nettbrett
    til brukeren.
  \item
    En egenutviklet hub kan være en del av en slik fremtid.
  \end{itemize}
\end{itemize}

\section{Intervju med SINTEF}\label{intervju-med-sintef}

\begin{itemize}
\tightlist
\item
  Tilstede: Anita Das og Kristine Holbø
\item
  Dato: mandag 8. mai kl. 10.30
\item
  Sted: SINTEF Teknologi og samfunn på Lerkendal
\item
  Formål: Demonstrere og vise fram prototypen. Ha en samtale om
  avstandsoppfølging og hva som vil bli viktig i utviklingen av
  produkter for avstandsoppfølging i fremtiden.
\item
  Stil: Uformelt, ikke noe brukertest. Semistrukturert intervju der
  intervjueren hadde forberedt samtaleemner og noen spørsmål.
\item
  Innhold

  \begin{itemize}
  \tightlist
  \item
    Introduksjon av meg og prosjektet.
  \item
    Introduksjon av intervjuobjekter og hva de jobber med.
  \item
    Samtale rundt velferdsteknologi og avstandsoppfølging
  \item
    Visning av prototype
  \item
    Samtale om hva de synes om prototypen og hva som vil bli viktig for
    utviklingen av produkter for avstandsoppfølging i framtiden.
  \end{itemize}
\item
  Oppsummering og evaluering

  \begin{itemize}
  \tightlist
  \item
    Sarpsborg kommune har flere pasienter som bruker sensorer og en
    annen teknologileverandør.
  \item
    Brukerne er forskjellige. Noen har veldig stor tillit til å levere
    fra seg måledata, mens andre er veldig skeptiske til hele konseptet.
  \item
    Tilbakemeldingene på prosjektet i Trondheim kommune som helhet er
    veldig positive.
  \item
    Bluetooth: Snakket om sikkerhet i Bluetooth og viktigheten av å være
    raskt ute med ny teknologi. De var ikke klar over
    sikkerhetsproblematikken til Bluetooth 4.0 og at pulsoksymeteret
    bruker denne standarden.
  \item
    Prototype: Muligheten til å kunne melde inn dagsform (subjektiv) må
    være tilstede i løsningen. Det gir økt trygghet for brukeren.
  \item
    Prototype: Fingeravtrykksensor vil ikke fungere som
    autentiseringsløsning for alle brukere. Noen har dårlig
    blodgjennomstrømning og tykke fingre. Må finne på nye måter for å
    autentisere slik at det blir tofaktor (sertifikat + noe annet).
  \item
    Prototype: De likte også bedre at huben var for seg selv og at
    sensorene var løse.
  \item
    Fremtiden: Så for seg en løsning som flere moduler. Slik at man kan bygge
    sammen alt som flere klosser for å tilpasse det til forskjellige
    brukere. For eksempel en modul for å rapportere dagsform.
  \item
    Fremtiden: De hadde veldig stor tro på innrapportering av sensordata.
    Teknologioptimistiske. Mente det var fremtiden. Så for seg
    automatisk analyse av data og bedre toveiskommunikasjon som et
    positivt hjelpemiddel.
  \end{itemize}
\end{itemize}
