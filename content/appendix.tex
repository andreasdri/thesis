%This is an Appendix
%%=========================================

\chapter{Kode}

\section{Hovedapplikasjon (smart-pulse-oximeter)}
\label{appendix:smart-pulse-oximeter}
Link til Github-repo med hele applikasjonen:
    \url{https://github.com/andybb/smart-pulse-oximeter}.

\begin{lstlisting}[frame=single, language=JavaScript,
    caption=smart-pulse-oximeter: index.js, label=lst:smart-pulse-oximeter]
const PulseOximeter = require('nonin-3230-ble');
const awsIot = require('aws-iot-device-sdk');
const { Button, RGBLed, Led } = require('pigpio-components');
const { enrollFingerAndRetrieveTemplate, setFingerprintTemplateAndVerify,
  blinkingFingerprintLed } = require('./fingerprint');
const { awsConfig } = require('./config');
const { delay } = require('./utils');
const logger = require('./logger');

const NUMBER_OF_SENSOR_READINGS = 20;

/* Noble attaches a listener that interrupts the GPIO cleanup.
 * The Noble listener does not do anything else than
 * invoking process.exit(1) if it is the last listener, */
process.removeListener('SIGINT', process.listeners('SIGINT')[0]);

const device = awsIot.device(awsConfig);
const button = new Button({ gpio: 24, isPullup: true });
const leftLed = new RGBLed({ red: 25, green: 8, blue: 7 });
const rightLed = new RGBLed({ red: 17, green: 27, blue: 22 });
const oximeterLed = new Led(12);

device
  .on('connect', () => {
    logger.log('debug', 'Connected to AWS');
    device.subscribe('oximetry');
    leftLed.color('yellow').strobe(2000);
    rightLed.stop();
  })
  .on('error', (err) => logger.error(err))
  .on('message', (topic, payload) => logger.log('debug', 'Received message', topic, payload));


const showErrorLeds = () => {
  leftLed.stop().color('red').on();
  rightLed.stop().color('red').on();
  delay(4000).then(() => {
    rightLed.stop();
    leftLed.stop().color('orange').strobe(2000);
  });
};

const startMeasurement = () => {
  logger.log('debug', 'Start measurement');
  leftLed.color('green').stop().on();
  setFingerprintTemplateAndVerify()
    .then(() => {
      logger.log('debug', 'Fingerprint verified');
      rightLed.stop().color('green').on();
      oximeterLed.setValue(255);
      let timer;
      let counter = 0;

      const onDiscover = ((pulseOximeter) => {
        logger.log('debug', 'Pulse oximeter discovered');
        pulseOximeter.connectAndSetup((error) => {
          if (error) {
            logger.error(error);
          }
        });

        pulseOximeter
          .on('data', (data) => {
            logger.log('info', 'Pulse oximeter data', data);
            counter += 1;
            device.publish('oximetry', JSON.stringify(data));
            if (counter === NUMBER_OF_SENSOR_READINGS) {
              logger.log('debug', 'Finished measurement. Cleaning up');
              leftLed.stop().color('yellow').strobe(2000);
              rightLed.stop();
              oximeterLed.off();
              clearTimeout(timer);
              pulseOximeter.completeMeasurement();
            }
          })
          .on('disconnect', () => {
            logger.log('debug', 'Pulse oximeter disconnect');
            if (counter < NUMBER_OF_SENSOR_READINGS) {
              logger.log('debug', 'Pulse oximeter disconnect before completion');
              oximeterLed.off();
              showErrorLeds();
              clearTimeout(timer);
            }
          });

      });

      PulseOximeter.discover(onDiscover);

      timer = setTimeout(() => {
        if (counter < NUMBER_OF_SENSOR_READINGS) {
          logger.log('debug', 'Pulse oximeter disconnect before completion');
          oximeterLed.off();
          rightLed.off();
          leftLed.color('yellow').strobe(2000);
        }
        logger.log('debug', 'Pulse oximeter: Stopping discover');
        PulseOximeter.stopDiscover(onDiscover);
      }, 120000);

    },
    (error) => {
      logger.error(error);
      showErrorLeds()
    });
};

button.on('click', () => {
  startMeasurement();
});

if (process.env.NODE_ENV === "development") {
  logger.log('debug', 'Environment: development');

  button.on('long press', () => {
    logger.log('debug', 'Button: Long press');
    leftLed.color('green').on();
    enrollFingerAndRetrieveTemplate()
      .then(() => {
        logger.log('debug', 'Fingerprint stored');
        rightLed.stop().color('green').on();
        delay(4000).then(() => {
          leftLed.color('yellow').strobe(2000);
          rightLed.stop();
        });
      }, (error) => {
        logger.error(error);
        showErrorLeds()
      }
      );
  });
}
\end{lstlisting}

\section{pigpio-components}
\label{appendix:pigpio-components}
Link til Github-repo med kode: \url{https://github.com/andybb/pigpio-components}.

\begin{lstlisting}[frame=single, language=JavaScript,
    caption=pigpio-components: Button class, label=lst:pigpio-components-button]
const EventEmitter = require('events');
const { Gpio } = require('pigpio');

class Button extends EventEmitter {
  constructor(options) {
    super();
    this._button = new Gpio(options.gpio, {
      mode: Gpio.INPUT,
      pullUpDown: options.isPullup ? Gpio.PUD_UP : Gpio.PUD_DOWN,
      edge: Gpio.EITHER_EDGE
    });
    this._timer = null;
    this._onInterrupt();
  }

  _onInterrupt() {
    let intervals = 0;
    this._button.on('interrupt', (value) => {
      if (value === 0) {
        this._timer = setInterval(() => {
          intervals++;
        }, 500);
      }
      else if (value === 1 && intervals === 0) {
        clearInterval(this._timer);
        intervals = 0;
        this.emit('click');
      }
      else if (value === 1 && intervals >= 4) {
        clearInterval(this._timer);
        intervals = 0;
        this.emit('long press');
      }
      else {
        if (this._timer) {
          clearInterval(this._timer);
          intervals = 0;
        }
      }

    });
  }
};

module.exports = Button;

\end{lstlisting}

\begin{lstlisting}[frame=single, language=JavaScript,
    caption=pigpio-components: Led classes, label=lst:pigpio-components-led]
const { Gpio } = require('pigpio');
const Color = require('color');

class Led {
  constructor(gpio) {
    this._led = new Gpio(gpio, { mode: Gpio.OUTPUT });
    process.on('SIGINT', () => {
      this.off();
      setTimeout(() => process.exit(1), 100);
    });
  }
  
  setValue(val) {
    if (val >= 0 && val < 256) {
      this._led.pwmWrite(val);
    }
    else {
      throw new Error('Value must be between 0 and 255');
    }
  }

  on() {
    this.setValue(255);
  }

  off() {
    this._led.digitalWrite(0);
  }
};

class RGBLed {
  constructor(ports) {
    this._leds =
      [new Led(ports.red), new Led(ports.green), new Led(ports.blue)];
    this._timer = null;
    this._rgb = Color.rgb(255, 255, 255).rgb().array();
    process.on('SIGINT', () => {
      this.stop();
      // Wait until other LEDs are cleaned up
      setTimeout(() => process.exit(1), 100);
    });
  }

  color(color) {
    this._rgb = Color(color).rgb().array();
    return this;
  }

  on() {
    this._leds.forEach((led, index) => led.setValue(this._rgb[index]));
    return this;
  }

  off() {
    this._leds.forEach((led) => led.off());
    return this;
  }
  
  stop() {
    this.off();
    if (this._timer) {
      clearInterval(this._timer);
    }
    return this;
  }

  strobe(interval = 1000) {
    if (this._timer) {
      clearInterval(this._timer);
    }

    let on = true; 
    this.on();
    this._timer = setInterval(() => {
      if (on) {
        this.off();
        on = false;
      }
      else {
        this.on();
        on = true;
      }
    }, interval);
    return this;
  }

  pulse(speed = 100) {
    if (this._timer) {
      clearInterval(this._timer);
    }

    let r = 0, g = 0, b = 0;
    this._timer = setInterval(() => {
      this._leds[0].setValue(r);
      this._leds[1].setValue(g);
      this._leds[2].setValue(b);
      r += 5;
      g += 7;
      b += 9;
      if (r > 255) {
        r -= 255;
      }
      if (g > 255) {
        g -= 255;
      }
      if (b > 255) {
        b -= 255;
      }
    }, speed);
    return this;
  }

  rainbow(speed = 500) {
    if (this._timer) {
      clearInterval(this._timer);
    }
    const colors = ['red', 'orange', 'yellow', 'green', 'blue', 'indigo', 'violet'];
    let index = 0;

    this._timer = setInterval(() => {
      this.color(colors[index]);
      this.on();
      index++;
      if (index > colors.length - 1) {
        index = 0;
      }
    }, speed);
    return this;
  }
}

module.exports = { RGBLed, Led };

\end{lstlisting}

\section{nonin-3230-ble}
Link til Github-repo med kode: \url{https://github.com/andybb/nonin-3230-ble}.

\begin{lstlisting}[frame=single, language=JavaScript,
    caption=Nonin 3230 noble-device library, label=lst:nonin-3230-library]
const NobleDevice = require('noble-device');

const SERVICE_UUID = '46a970e00d5f11e28b5e0002a5d5c51b';
const NOTIFY_CHAR  = '0aad7ea00d6011e28e3c0002a5d5c51b';

const Nonin3230 = function(peripheral)  {
  NobleDevice.call(this, peripheral);
};

Nonin3230.SCAN_UUIDS = [SERVICE_UUID];

Nonin3230.is = (peripheral) => (
    peripheral.advertisement.localName.indexOf('Nonin3230_') > -1
);

NobleDevice.Util.inherits(Nonin3230, NobleDevice);

NobleDevice.Util.mixin(Nonin3230, NobleDevice.BatteryService);
NobleDevice.Util.mixin(Nonin3230, NobleDevice.DeviceInformationService);

Nonin3230.prototype.onMeasurement = function(data) {
  const b = data[1];
  const status = {
    encryption: !!(b & 0x40),
    lowBattery: !!(b & 0x20),
    correctCheck: !!(b & 0x10),
    searching: !!(b & 0x8),
    smartPoint: !!(b & 0x4)
  }
  const counter = data.readInt16BE(5);
  const oxygenSaturation = data.readInt8(7);
  const pulseRate = data.readInt16BE(8);
  this.emit('data', { counter, oxygenSaturation, pulseRate, status });
};

Nonin3230.prototype.completeMeasurement = function(done) {
  this.writeDataCharacteristic(SERVICE_UUID, '1447af800d6011e288b60002a5d5c51b',
    new Buffer([0x62, 0x4E, 0x4D, 0x49]), done);
};

Nonin3230.prototype.connectAndSetup = function(callback) {
  NobleDevice.prototype.connectAndSetup.call(this, function(error) {
    this.notifyCharacteristic(SERVICE_UUID, NOTIFY_CHAR, true,
      this.onMeasurement.bind(this), function(err) {
        callback(err);
      });
  }.bind(this));
};

module.exports = Nonin3230;

\end{lstlisting}


\chapter{Formell henvendelse til Trondheim kommune}
\label{appendix:formell}
\textit{Merk: Henvendelsen er omskrevet tre steder for å anonymisere}\newline
Til Prosjekt- og programledelsen Velferdsteknologi, Trondheim kommune

Vi søker med dette om samarbeid for en forskningsprosjekt for en masterstudent (Andreas Drivenes) ved Institutt for Datateknikk og Informatikk ved
NTNU innen området veldferdsteknologi med undertegnede som veileder. Prosjektet omhandler forbedring av brukervennlighet for pulsoksimeter til
avstandsoppfølging av kronisk syke. Bakgrunnen for prosjektet er et mangeårig samarbeid med SINTEF Helse (omskrevet) rundt brukervennlighet og
design av veldferdsteknologi.

Studenten har koblet opp et Nonin Pulsoksiometer mot en skytjeneste via
Bluetooth (omskrevet). Bakgrunnen for prosjektet er en mistanke om at en del kronisk syke vil ha problemer med tekniske løsninger for avstandsoppfølging dersom
de selv alene skal bruke måleutstyr som krever teknisk kunnskap, f.eks. oppkobling av bluetooth til nettbrett. Den nye s.k. Internet-of-Things
teknologien med direkte kobling til skytjenester gjør det mulig å ha måleutstyret direkte på internett, slik at man slipper å gå veien om f.eks. et
nettbrett. Vi tror dette vil kunne bedre brukervennligheten for kronisk syke, og gi en bedre tjenestekvalitet. Det vil muligens også kunne redusere
behovet for nettbrett for noen brukere.

Rent konkret så ser vi for oss følgende samarbeid med kommunen:

\begin{enumerate}
\def\labelenumi{\arabic{enumi}.}
\tightlist
\item
    Samarbeid med C.C. i form av (omskrevet):

  \begin{itemize}
  \tightlist
  \item
    Ett intervju a maks to timer (mars)
  \item
    Feedback på teknisk løsning, maks to timer (april).
  \end{itemize}
\end{enumerate}

Hvis mulig ønsker vi også å få tilbakemelding fra helsearbeidere som er
nærmere feltet:

\begin{enumerate}
\def\labelenumi{\arabic{enumi}.}
\setcounter{enumi}{1}
\tightlist
\item
  Fokusgruppe med 4-5 helsearbeidere som har erfaring med
  avstandsoppfølging av kronisk syke.

  \begin{itemize}
  \tightlist
  \item
    Hjelp til rekruttering av fokusgruppedeltagere.
  \item
    Frikjøp av helsearbeidere, maks en time. (april).
  \end{itemize}
\end{enumerate}

Med håp om konstruktivt samarbeid.

mvh
Dag Svanæs

\begin{verbatim}
__________________________________________________
Dag Svanæs, Professor, PhD
Department of Computer Science (IDI)
Norwegian University of Science and Technology (NTNU)
Trondheim, Norway
\end{verbatim}

\chapter{Henvendelse til Nonin om pulsoksymeter}
\label{appendix:pulsoksymeter}
\textbf{Question about Nonin 3230 Bluetooth integration}
(Sendt til \url{oem@nonin.com})

Hi!

I am currently writing a master's thesis about welfare technology at the Norwegian University of Science and Technology. 

This is done in collaboration with the municipality of Trondheim which uses Nonin 3230 in their
solution for remote monitoring of patients with chronic diseases. 

As a part of the thesis, I am going to implement a simple prototype of how a smart and cloud connected pulse oximeter could work.

Would it be possible to receive the technical documentation for Nonin 3230?
I am especially interested in hearing about how the GATT-based proprietary profile works, so
that I can make sense of the data received from the sensor. 

Best regards,\newline
Andreas Drivenes\newline

\chapter{Invitasjoner til intervju}
\label{appendix:invitasjon_evaluering}

\section{Intervjuhenvendelse til Trondheim kommune (case-studie)}
Hei!

Har dere mulighet til å ta et intervju torsdag eller fredag i neste uke? Når som helst på dagen passer. 
Ellers så passer det når som helst på dagtid uka etter det igjen (27. mars - 31. mars).

Det hadde vært flott om jeg fikk lov til å ta opp intervjuet. Jeg kommer til å forberede noen skriftlige spørsmål
(som jeg kan sende på forhånd om dere ønsker),
men vi kan gjerne bevege oss litt utover det med oppfølgingsspørsmål om det passer seg sånn. 

Det hadde vært fint å titte kort på løsningen dere har nå etter intervjuet, og ta noen bilder av den.

Andreas

\section{Intervjuhenvendelse til Trondheim kommune og SINTEF (evaluering og oppsummering)}
Hei! 

Jeg har utviklet en prototype dere kan se på og teste, og som jeg gjerne vil ha tilbakemeldinger på.

Etter testen tenkte jeg at vi kunne snakke litt om hvordan brukere har opplevd løsningen som Trondheim kommune prøver nå, hva dere tenker om smarte
og integrerte sensorer som alltid er på nett, og hva som vil være viktig når det skal utvikles smarte velferdsteknologiprodukter for oppfølging av
kronisk syke i fremtiden.

Har dere mulighet til å møte meg i slutten av neste uke, eller tidlig uken etter det igjen (tidsrommene 26. - 28. april og 2. - 5. mai)? Jeg anslår
at det kommer til å ta en time til halvannen time, og jeg er veldig fleksibel på tidspunkt på dagtid. 

Vi kan godt møtes hos dere, Z.Z., hvis det er enklest for dere.

Andreas

\section{Intervjuhenvendelse til teknologileverandøren Imatis}
Hei! 

Mitt navn er Andreas Drivenes, og jeg går siste året på datateknologi på NTNU. Jeg skriver for tiden en masteroppgave
om avstandsoppfølging av kronisk syke i samarbeid med Trondheim kommune og SINTEF. Veilederen min er professor Dag Svanæs.

Jeg har fått låne et Nonin 3230-pulsoksimeter av Trondheim kommune som jeg har integrert med en Raspberry Pi Zero.
Den har en fingeravtrykkssensor, en knapp og to leds. Den publiserer sensordataen til AWS IoT, og jeg har satt opp en helt
enkel grafisk applikasjon som viser målingene. Koden til oppgaven kommer til å ligge åpent på Github.

Det ville være til stor hjelp for oppgaven om dere kunne snakke litt om hva slags teknologi dere bruker i løsningen deres i dag.

Rent konkret lurer jeg på om du (eller noen andre) kunne være med på et kort intervju på appear.in eller lignende? 
Jeg ser for meg at det kommer til å ta maks 30-45 minutter. Temaer for samtalen kommer til å være hvilken teknologi og arkitektur
dere bruker i løsningen for Trondheim kommune i dag, sikkerhet, tanker om prototypen og teknologien jeg har brukt, og den
digitale utviklingen i velferdsteknologi og avstandsoppfølging av kronisk syke i årene som kommer.

Jeg er ledig på dagtid det meste av den neste uken.

Jeg sender selvsagt en kopi av masteroppgaven når den blir ferdig i midten av juni.

\verb|--| \newline
Med vennlig hilsen \newline
Andreas Drivenes \newline

\chapter{Forberedelse til intervju med Trondheim kommune}
\label{appendix:forberedelse}
\textbf{Info før intervjuet:}
Tar opp samtalen på mobilen, sletter opptaket etterpå når jeg er ferdig. 
 
Jeg tenkte jeg skulle starte med å snakke litt kort om meg og hva jeg holder på med og hva planene er for masteroppgaven. Jeg heter Andreas Drivenes og går i femte klasse
datateknologi på NTNU. Jeg skriver en masteroppgave om avstandsoppfølging av kronisk syke med Dag Svanæs som veileder. Opprinnelig var temaet for oppgaven min å undersøke
ulike skybaserte tingenes internett-løsninger, og casen vi hadde sett på var deling av veidata for rullatorer. Imidlertid hadde jeg et møte med X.X. og Y.Y på Sintef som pekte meg i retningen av avstandsoppfølging av kronisk syke siden det var noe som faktisk var brukt i praksis og som det var tester på i Trondheim. Så i fjor høst skrev jeg en prosjektoppgave om avstandsoppfølging og hvordan det kunne bli brukt sammen med moderne skyløsninger. Jeg skal snakke litt mer om hva vi tenker å lage som prototype etterpå
 
Velkommen til intervju, det var veldig flott at dere kunne sette av tid til det. Først snakke litt om dere og hva dere jobber med. Så litt kort om bakgrunnen til velferdsteknologiprogrammet til Trondheim, deretter om løsningen dere har for avstandsoppfølging av kronisk syke i dag - hvordan den fungerer fra A-Å, det som er bra med den dårlig med den etc, hvordan dere tror det vil være i fremtiden, hvordan brukere ser på den osv. Etter det tenkte jeg å presentere litt av min egen løsing og spørre rundt den. Tilslutt tenkte jeg at vi kunne se litt på løsningen deres om jeg får lov til det
 
\textbf{BAKGRUNN}
Kan dere starte med å si hva dere heter, og hva dere jobber med i Trondheim Kommune?
 
Jeg har inntrykk av at Trondheim kommune er veldig flinke på å ta i bruk ny teknologi i helsetjenester. Kan dere si litt om bakgrunnen for det og hvilke prosjekter dere har nå?
 
Hvordan foregår samarbeidet med Helsedirektoratet? Utveksler dere kunnskap med andre kommuner?
 
\textbf{HELSAMI+}
Spørsmål om avstandsoppfølging av kronisk syke generelt:
Hvor god klinisk vurdering kan man gjøre av helsetilstand med måling av blodtrykk, vekt, puls og oksygenmetning, samt spørsmål om egen dagsform?
 
Hvorfor er det så viktig at pasienter følges opp hjemme? Hva er fordelen med det? (trygghets- og mestringsfølelse, færre reinleggelser?)
 
Hva er den historiske bakgrunnen for avstandsoppfølging-prosjektet (altså HelsaMi+) til Trondheim kommune? 
 
Kan dere beskrive hvordan løsningen i HelsaMi+ fungerer for en typisk pasient fra A til Å? Fra installering av utstyr, til måling, til individuell oppfølgingsavtale?
 
Hva gjøres med målingen som kommer inn? Har helsepersonell tilgang på den? Hva er det som eventuelt stopper at helsepersonell kan se direkte på målingene?
 
Kan pasienten selv se målingene sine i nettbrettet? Sammenlignet med historiske data.
 
Hva slags type spørsmål er det pasienten må svare på i løsningen? (Det som kalles for fastlagt spørreskjema)
 
Er det pasienten selv som må sørge for at alt har batteri og strøm og koblet opp på Internett? Har det vært noen utfordringer der?
 
Hvordan bestemte dere hvordan løsningen skulle se ut? Hvem var med på å utforme kravene til løsningen? 
 
Hva er tilbakemeldingen fra pasientene på løsningen? Har dere gjennomført noen brukertester med eldre pasienter? Har pasientene vært med på utviklingen av tjenesten?
 
Hva er tilbakemeldingen fra helsepersonell, altså folk i kommunen hos dere, fastleger osv. på løsningen?
 
Er det noe regelverk eller sikkerhetsutfordringer som har stoppet dere i fra å gjøre løsningen på en annen måte?
 
Hva slags type nettbrett er det brukerne får?
 
Hvor mye koster utplassering av systemet deres hos en pasient? (ca.-pris, hvis jeg kan spørre om det)
 
Hvordan har samarbeidet deres vært med Imatis? 
Har dere kommet med innspill som har gjort at de har endret på ting? 
 
“Da prosjektet følges av en forskningspartner så vil du for å delta måtte signere et samtykkeskjema for at du kan bli intervjuet og at anonymiserte data kan hentes ut for å se på om denne tjenesten gir en bedre effekt på for eksempelvis færre reinnleggelser eller en økt mestring og trygghetsfølelse for brukerne.”
Vil denne forskningen bli publisert offentlig? Er det allerede nå blitt publisert noe? Når tenker man å publisere? Kan jeg kontakte noen for å få tilgang til noe materiale? 
 
\textbf{HELSAMI+ i framtiden}
Sitat: “Prosjektet vil i løpet av 2017 videreutvikle tjenesten til å rette et større fokus på forebyggende tiltak som bruker kan gjennomføre selv med bistand fra teknologien.”
Hva betyr dette?
 
Hvis dere kunne endre på noe i løsningen selv, hva ville dere endret på da?
 
Hva er planene for HelsaMi+ framover? Hva vil skje etter 2018?
 
Kunne det vært et dataprogram som automatisk oppdaget avvik basert på pasientenes historiske data? 
 
Integrering med andre løsninger? 
 
\textbf{MIN LØSNING}
smart pulsoksymeter med fingerprintsensor som autentiseringsmetode (slipper styret med passord)
trykke på knapp -> blinker gult -> autentiserer med finger -> grønt lys, blinker gult lengre borte -> måler oksygenmetning og puls -> blinker grønt når ferdig.
 
Integrert løsning med autentisering, toveiskommunikasjon til skyen, lager prototype som skal testes, brukeren forholder seg til én dings
 
Hva tenker dere rundt en sånn løsning?
 
Lurt med fingerprintsensor som autentisering, eller potensielle fallgruver? 
 
Kan det være bra å få info om når målingene skal gjøres?
 
\textbf{TIL SLUTT}
Tusen takk for at dere stilte.
Er det noe jeg burde spurt om som jeg har glemt?


\chapter{Samtykkeerklæringer}
\label{appendix:samtykke}

\section{Anonym}
\begin{center} \textit{Tingenes internett for avstandsoppfølging av kronisk syke \newline
Deltakelse på gruppeintervju}
\end{center}
\underline{Samtykkeerklæring} \newline
Jeg har mottatt muntlig informasjon om studien, og fått anledning til å stille spørsmål. Jeg er klar over at det er frivillig
å delta, og at jeg kan trekke meg fra studien når som helst uten å oppgi noen grunn. Jeg samtykker i å delta i studien.

Det vil bli tatt lydopptak.  Dette gjøres for at vi skal kunne vurdere intervjuene i etterkant og for å sikre oss at
vi har forstått utsagn og handlinger riktig. Vi vil sørge for at materiale vil bli anonymisert slik at det ikke vil
være mulig å føre opplysningene tilbake til enkeltpersonene som deltar i prosjektet. Dette innbærer at informasjon som
blir formidlet til offentligheten ikke vil kunne settes i sammenheng med den enkelte. Det er kun de involverte i prosjektet
som vil kunne se opptakene i ettertid.

Trondheim, \verb|_________________________|

\verb|_________________________|\newline
Navn

\verb|_________________________|\newline
Underskrift \newline

Masterstudent ansvarlig for studien: \newline
	Andreas Drivenes, e-mail: andreas.drivenes@gmail.com \newline
Faglig ansvarlig: \newline
	Professor Dag Svanæs, e-mail: dags@idi.ntnu.no \newline
Ansvarlig avdeling: \newline
	Institutt for Datateknologi og Informatikk (IDI), NTNU

\section{Ikke-anonym}
\begin{center} \textit{Tingenes internett for avstandsoppfølging av kronisk syke \newline
Deltakelse på gruppeintervju}
\end{center}
\underline{Samtykkeerklæring} \newline
Jeg har mottatt muntlig informasjon om studien, og fått anledning til å stille spørsmål. Jeg er klar over at det er frivillig
å delta, og at jeg kan trekke meg fra studien når som helst uten å oppgi noen grunn. Jeg samtykker i å delta i studien.

Det vil bli tatt lydopptak.  Dette gjøres for at vi skal kunne vurdere intervjuene i etterkant og for å sikre oss at vi har
forstått utsagn og handlinger riktig. Jeg samtykker i at mitt navn kan bli brukt i masteroppgaven og at det jeg sier kobles
til mine utsagn. Jeg blitt informert om at jeg har fått mulighet til å lese igjennom referatene fra intervjuene og
kommentere på eventuelle feil i transkribering. 

Det er kun de involverte i prosjektet som vil kunne høre opptakene i ettertid.

Trondheim, \verb|_________________________|

\verb|_________________________|\newline
Navn

\verb|_________________________|\newline
Underskrift \newline

Masterstudent ansvarlig for studien: \newline
	Andreas Drivenes, e-mail: andreas.drivenes@gmail.com \newline
Faglig ansvarlig: \newline
	Professor Dag Svanæs, e-mail: dags@idi.ntnu.no \newline
Ansvarlig avdeling: \newline
	Institutt for Datateknologi og Informatikk (IDI), NTNU
