\chapter{Diskusjon}
\label{ch:discussion}

\section{Begrensninger med
prototypen}\label{begrensninger-med-prototypen}

\begin{itemize}
\tightlist
\item
  Lav grad av toveiskommunikasjon implementert.
\item
  Fingeravtrykk-template lagres direkte på sensoren, den burde blitt
  hentet fra sensoren og så lagret i skyen. Deretter burde den blitt
  hentet hver gang man trengte den.
\item
  Paring og bonding med Bluetooth. Enheten må pares hver gang i
  prototypen. Dette er en begrensning med det underliggende rammeverket
  som brukes.
\item
  I prototypen brukes MQTT over WebSockets til å sende data. Men det er
  såpass få målepunkter som sendes hver dag at man like gjerne kunne
  brukt HTTP REST med kryptering. Det er få fordeler med å bruke en
  protokoll som MQTT i dette prosjektet. Det hadde vært mer relevant om
  sensorene sendte data kontinuerlig hele døgnet.
\end{itemize}

\section{Sikkerhet}\label{sikkerhet}

\begin{itemize}
\tightlist
\item
  Tofaktorautentisering
\item
  Hva oppfyller kravene til tofaktorautentisering? Oppfyller
  klientsertifikat og enhet utplassert i hjem sammen med
  fingeravtrykksensor kravene?
\item
  Hvilke alternativer finnes det til å bruke fingeravtrykkssensor som
  autentisering? Både med tanke på brukbarhet og sikkerhet.
\item
  Bluetooth
\item
  BLE har flere sikkerhetsproblemer, spesielt i versjon 4.0 og 4.1. Skal
  man gjøre en oppfordring om kun å bruke Bluetooth 4.2? Hvilke
  avveininger og anbefalinger burde man gjøre dersom man må bruke en
  enhet med 4.0 eller 4.1? Det tar lang tid å ta i bruk en ny
  Bluetooth-standard. Produsenter av velferdsteknologi må følge med på
  den teknologiske utviklingen. Problemer her: tar lang tid å få et
  produkt godkjent, koster sikkert masse penger.
\end{itemize}

\section{Personvern og datalagring}\label{personvern-og-datalagring}

\begin{itemize}
\tightlist
\item
  Krav om at norske helsedata skal ligge i Norge. Dermed kan man ikke
  bruke en eksisterende skyløsning som AWS. AWS lagrer data i Europa.
  Det fører til at mye av infrastrukturen må lages fra bunnen av. Snakk
  om at Azure skal få til en løsning der man har hele økosystemet
  kjørende i Norge.
\item
  Ny personvernforordning gir flere rettigheter til forbrukere. Det
  gjelder blant annet kontroll på egen data, oversikt over hva den
  brukes til, mulighet til å slette den osv. Man må kunne argumentere
  for at dataen som lagres er nyttig for forbrukeren. Er da å lagre all
  sensordataen nyttig? Kan man argumentere for det?
\end{itemize}

\section{Frittstående løsninger kontra nettbrett, mobil,
PC}\label{frittstuxe5ende-luxf8sninger-kontra-nettbrett-mobil-pc}

Diskusjon og drøfting på bakgrunn av evalueringen om hvor vellykket det
var å implementere en frittstående løsning. Hvilke fordeler og ulemper
har man med å ta i bruk et eksisterende økosystem? Stikkord:
sikkerhetsoppdateringer, vedlikehold, innkjøpskostnad, hva har brukere
hjemme osv.

\section{Metodediskusjon}\label{metodediskusjon}

\begin{itemize}
\tightlist
\item
  Diskutere forskningsmetodene som er brukt? Hvor stor troverdighet kan
  man ha til forskningen? Kunne noe vært gjort annerledes?
\end{itemize}

