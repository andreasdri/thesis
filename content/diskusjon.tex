\chapter{Diskusjon}
\label{ch:discussion}

Det er arkitekturen og basisteknologien som er det viktige i dette forskningsprosjektet. Basisteknologien
ble satt inn i et praktisk problem i et komplekst domene. Sensorer til støtte for avstandsoppfølging er et
godt eksempel på noe som kan kobles til den nye arkitekturen for tingenes internett, som beskrevet av \citet{iot_harvard_smart}.

Prototypen ble laget
for å si noe om denne basisteknologien med forskningsstrategien design og kreasjon. Case-studien av avstandsoppfølging
i Trondheim kommune ble gjennomført for å øke kunnskapen om domenet og finne ut hvilke utfordringer kommunen sliter med.
Det viste seg at kommunen hadde problemer med å få til tofaktorautentisering eller noe med tilsvarende grad av sikkerhet. Kommunen
jobbet også med å få til sertifikater og distribusjon av sertifikater. Eksisterende skyteknologi har allerede noe av dette innebygget.
AWS IoT har infrastruktur for å håndtere sertifikater, selv om distribusjonen av sertifikatene
til enhetene må ordnes på egenhånd.

Prosjektet viste at det er enkelt å sette opp et standard grafsystem med målinger tilbake i tid på toppen av åpne kildekode-løsninger
på veldig kort tid. Kommunen var imponert over hvor langt prosjektet hadde kommet på en måned med utvikling.
Under evalueringen pekte kommunen på at de i dag har for dårlige verktøy til å se på måledata, og dermed
ikke kan nyttiggjøre seg sensorteknologien. Leverandørene av tjenester til bruk i velferdsteknologi har et ansvar for
å levere gode løsninger og ta i bruk ny teknologi, spesielt dersom de får være med på forsknings- og utviklingsprosjekt (FOU). Kommunen
har selv sagt i intervju at de ønsker å være fremoverlente på teknologi og til stede på de plattformene brukerne er.
Det er tydelig at de løsningene kommunen bruker i dag
ikke ser så moderne ut og bruker modifisert hyllevare fra leverandøren.

Det hadde vært ønskelig om prosjektet hadde kommet enda lengre med implementasjonen av en større skyløsning med
toveiskommunikasjon og mer tilbakemeldinger til brukeren for å løse flere av utfordringene kommunen og SINTEF pekte på.
Dette kunne dannet et enda større bilde av egnetheten
til skybasert IoT som teknologisk plattform for avstandsoppfølging. Allikevel er erfaringene at arkitekturen for sensoren fungerte
veldig bra og at det var enkelt å komme i gang med. Dokumentasjonen til skyteknologien var god. Prosjektet sitter igjen med at
plattformen var til å stole på og har høy tillit til at det fungerer.

MQTT ble brukt som kommunikasjonsprotokoll mellom prototype og skyløsning.
På en annen side kunne kanskje en mer tradisjonell løsning også fungert.
MQTT-protokollen egner seg kanskje enda bedre for sensorer
som sender data kontinuerlig. Brukeren sender 20 sekunder med data hver dag i bruksscenarioet som ble skissert i dette prosjektet. Det er ikke så veldig
mye data. Den samme dataen kunne blitt sendt over HTTPS og en standard arkitektur som REST istedenfor.
Istedenfor sertifikater for autentisering kunne en brukt tokens med så og så lang gyldighet som bevis på at enheten er logget inn.

I fremtiden kan det hende at det blir mer aktuelt å gå med sensorer over lengre tidsperioder dersom det blir vurdert som klinisk relevant
for en bruker. I så fall vil AWS IoT være veldig skalerbar og pålitelig for å håndtere en økt mengde meldinger. Verdien av
automatisk skalerbarhet, en lang rekke med mulige integrasjoner og at man slipper å ha egen infrastruktur må ikke undervurderes.

Selv om AWS IoT har god nytteverdi mellom klient og server, har ikke skyplattformen noe med kommunikasjonen fra sensor til klient. Her er det
det Bluetooth-teknologien som råder og kommunen er prisgitt at leverandørene er tidlig ute med ny teknologi. I praksis vil nok ikke
sikkerhetsrisikoen ved bruk av Bluetooth være så stor. Angriperne må være helt i nærheten for at det skal være teoretisk mulig å sette i gang
et angrep.

\section{Metodediskusjon}
Starten av diskusjonskapittelet trakk mest fram det positive med valgene av metodikk i dette prosjektet.
Det hadde vært nyttig med mer teknisk informasjon om hvordan HelsaMi+ fungerer fra leverandøren for å sammenligne,
se hvor langt de har kommet sikkerhetsmessig og hva de tenker om fremtiden. Leverandøren Imatis ble kontaktet for et
intervju, men de svarte ikke på henvendelsen. Kanskje mer burde blitt gjort for å snakke med en leverandør av tekniske løsninger
til dette prosjektet. Det er en viktig komponent som mangler litt. Det er en annen leverandør i Oslo og Sarpsborg.
De to leverandørene kunne blitt sammenlignet med tanke på hva de tilbyr av tjenester og hvordan de har bygget opp sin løsning
mot avstandsoppfølging. Det kunne vært interessant å høre hva leverandørene tenker om denne arkitekturen for avstandsoppfølging,
hva de identifiserer som de største utfordringene for en slik arkitektur og om de planlegger å gå over på en lignende
arkitektur selv. Dette er noe som kan jobbes videre med i andre forskningsprosjekt. Mulighetene er absolutt til stede
for å sette opp en lignende arkitektur fra bunnen av selv med egen infrastruktur.

Hele temaet er veldig stort, og det hadde kanskje vært
lurere å snevret inn hvilke kvalitetskrav man skulle sette ekstra søkelys på, og hvordan disse kravene
skulle vært testet slik at en kunne gjort et bedre dypdykk
og snakket med eksperter innenfor det området. På den måten kunne en for eksempel sett nærmere på problematikken rundt
tofaktorautentisering og snakket med eksperter på det.

\section{Validitet}
\label{sec:validitet}

\subsection{Bekreftbarhet og pålitelighet}
Tidligere i kapittelet ble fremgangsmåten til studiet presentert. All kode ligger åpent tilgjengelig på Internett slik at hvem som helst kan
titte på den. Transkriberinger av intervjuene er tilgjengelige ved forespørsel. Tidligere ble implementasjonen av prototypen beskrevet i
så god detalj at det er mulig å lage noe tilsvarende kun ved å lese i teksten og se på koden.

\subsection{Troverdighet}
Alle intervjuene med Trondheim kommune må tolkes med det faktum at prosjektledere og ansatte har en tilknytning som
gjør at de ønsker at avstandsoppfølgingprosjektet
skal lykkes i bakhodet. Aktører direkte involvert i et prosjekt er mottakelige for bekreftelsestendenser der man søker etter å bekrefte noe
man allerede tror. Selv om SINTEF er en tredjepart som bistår Trondheim kommune, er de også mye involvert i prosjektet.
Det er mulig at forskningen kunne blitt enda mer troverdig med en kritisk tredjepart uten direkte tilknytning til velferdsteknologiprosjektet i Trondheim --
gjerne en stemme som ikke er udelt positiv til avstandsoppfølging som teknologi for velferd.
Fordelen med å snakke med personer som jobber med avstandsoppfølging i Trondheim kommune, er at de har veldig god kunnskap om temaet
og muligheten til å sammenligne løsningen som presenteres i denne forskningen med noe som eksisterer i praksis i dag.
De er en troverdig stemme i avstandsoppfølgingdebatten.

I fokusgrupper, brukbarhetstester og intervjuer har deltakerne en tendens til å ville blidgjøre
den som har laget produktet (i dette tilfellet utvikleren), dersom det er samme person som
gjennomfører testene. Dette er en veldig menneskelig egenskap. SINTEF var en uavhengig tredjepart i intervjuer
med brukere av HelsaMi+ for å få brukerne til å snakke mer fritt. I dette prosjektet har samme person utviklet og gjennomført
intervjuene, men det virker som intervjudeltakerne har snakket veldig fritt og vært åpne i samtalene.

Transkriberingen av store deler av de tre gjennomførte intervjuene ble gjort for å øke troverdigheten og motvirke egen
tendens til å tolke alt på en spesiell måte. Det gjorde at man fikk konteksten til hele intervjuet. Et spørsmål
er uansett om man med utviklerblikket som er perspektivet i dette prosjektet er i stand til å tolke alt som blir sagt.
Noe av det må nødvendigvis bli subjektivt.

\subsection{Overførbarhet}
Spørsmålet som må stilles er om funnene og arbeidet gjort i dette prosjektet kan overføres til avstandsoppfølging som generelt tilfelle.
Trondheim kommune brukes som eksempel, og konklusjonen vil generalisere med kun dette tilfellet som bakgrunn.
Er det faktisk mulig å finne noen implikasjoner for utviklingen av skybasert \gls{iot} som teknologisk plattform for avstandsoppfølging?
Hvilke resonnementer og argumenter er i så fall disse resonnementene fundert i?

Trondheim kommune har ikke så mange brukere med sensorer, og SINTEF trakk fram at andre kommuner har kommet lengre på det området. Uansett
vil det være rimelig å anta at andre kommuner og leverandører vil ha noe av den samme problematikken som Trondheim med sikkerhet, sertfikater, autentisering,
personvern og skalering.

\section{Avgrensning av forskningen}
Flere av utfordringene som ble identifisert i dette prosjektet kunne vært tema for andre og mer detaljerte undersøkelser. Det ble allikevel
vurdert som nyttig å ta med helheten i avstandsoppfølgingsprosjektet, selv om kanskje ikke alt var direkte relevant for forskningsspørsmålene.
I virkeligheten er alt komplekst, og mange ting spiller inn på arkitekturen til en løsning. Det er krav fra flere forskjellige hold som en må ta hensyn til
og forholde seg til.

Det har vært et bevisst valg å distansere seg fra flere av disse kravene for å ikke drukne i informasjon, slik at søkelyset kunne ligge på IoT som
teknologisk plattform med arkitektur og programvareutvikling. Flere ting kommunen og SINTEF nevner kunne blitt nøyere undersøkt,
men her har bare disse instansene vært sannhetsvitner for problematikk
som dukker opp i praksis når dette skal ut i produksjon. Selv om ikke brukbarhet har vært viktig for prosjektet, er tilbakemeldingene på prototypen når det gjelder
brukskvalitet tatt med allikevel for å få med det store bildet.

Slik det ble forklart i delkapittel \ref{sec:avgrensning}, har ikke de kliniske vurderingene man kan gjøre med sensordataene vært drøftet
i prosjektet.
