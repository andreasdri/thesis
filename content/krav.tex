\chapter{Krav til løsning}
\label{ch:requirements}

\section{Funksjonelle krav}

\subsection{Krav til prototype}

\begin{enumerate}
    \item [\textbf{FK1}] Når brukeren trykker på knappen skal systemet sette i gang en måling
        \begin{enumerate}
          \item [\textbf{FK1.1}] Brukeren må legge fingeren på fingeravtrykksensoren for å autentisere innen 10 sekunder.
          \item [\textbf{FK1.2}] Ved vellykket autentisering skal brukeren få 120 sekunder til å gjennomføre
              en måling.
          \item [\textbf{FK1.3}] Brukeren starter målingen ved å stikke fingeren inn i pulsoksimeteret.
              En måling er definert som kontinuerlig datainnhenting i 20 sekunder.
          \item [\textbf{FK1.4}] Systemet skal sende data fra målingen kontinuerlig til en skytjeneste.
          \item [\textbf{FK1.5}] Når en måling er ferdig, skal skjermen på pulsoksimeteret og lysdiodene indikere til brukeren
              at det er godkjent.
          \item [\textbf{FK1.6}]Systemet skal vise rødt lys og sende brukeren tilbake til sovende modus dersom et av de
              foregående kravene ikke oppfylles.
        \end{enumerate}
    \item [\textbf{FK2}] Når brukeren holder inne knappen i over fire sekunder og slipper i utviklingsmodus,
        skal prosessen for å registrere et fingeravtrykk settes i gang.
        \begin{enumerate}
          \item [\textbf{FK2.1}] Registreringen blir gjort sammen med helsepersonell som forklarer hva som skal gjøres.
          \item [\textbf{FK2.2}] Dersom noe går galt under registreringen, skal systemet vise rødt lys og sende brukeren
              tilbake til sovende modus igjen.
        \end{enumerate}
\end{enumerate}

% øke vinduet for autentisering: mange mange minutter. tilbakemelding fra sintef

\section{Kvalitetskrav}

I følge \citet{softarch} er sikkerhet, ytelse og tilgjengelighet de viktigste kvalitetskravene for skyteknologi.
Personvern er viktig for kritiske helsedata, og er noe som Datatilsynet legger stor vekt på. I tillegg til disse kvalitetskravene
er interoperabilitet og brukervennlighet trukket fram som kvalitetskrav. Basert på bakgrunnsmaterialet og case-studien,
ble kvalitetskravene rangert for en frittstående skybasert velferdsteknologiløsning.

\begin{enumerate}
    \item Sikkerhet
    \item Personvern
    \item Interoperabilitet
    \item Tilgjengelighet
    \item Ytelse
    \item Brukervennlighet
\end{enumerate}

Disse kvalitetskravene er beskrevet nøyere i de neste delkapitlene.

\subsection{Sikkerhet}
Systemet skal være beskyttet mot \gls{mitm} og tyvlytting, og skal ha tilgangskontroll på alle ressurser.
Alle kommuniserende enheter skal være autentisert for å unngå "forging of data". % @TODO fix
Det skal være intern tilgangskontroll for utviklere, helsepersonell og tredjepartsløsninger.

\subsection{Personvern}
Datatilsynet foreslår at alle nye velferdsteknologisystemer lages med innebygd personvern (\textit{privacy by design}),
noe som vil si at det tas hensyn til personvern i alle fasene av utviklingsløpet \citep{datatilsynet_privacy}. Det er mye billigere
enn å endre systemet for øke personvernshensynet i etterkant.

For velferdsteknologi spesielt, oppgir \citet{datatilsynet_welfare} ti viktige generelle retningslinjer for å sikre personvernet
i utviklingen av ny velferdsteknologi:

\begin{enumerate}
    \item Velg den minst inngripende løsningen
    \item Begrens mengden data som lagres
    \item Velg sanntidsløsning hvis mulig
    \item Lagre lokalt hvis mulig
    \item La brukeren ha kontroll over løsningen
    \item Slett data etter bruk
    \item Begrens tilgangen til informasjon
    \item Innsyn i egne data
    \item Dataene bør krypteres
    \item Anonymisering av data
\end{enumerate}

\subsection{Interoperabilitet}
Det er forventet at andre og tidligere utviklede systemer må integreres i løsningen. Derfor trenger løsningen flere forskjellige
grensesnitt som gjør det mulig å kommunisere med andre plattformer. Norge har bestemt at Continua-rammeverket skal brukes i nye
velferdsteknologiløsninger.

\subsection{Tilgjengelighet}
I velferdsteknologi kan tilgjengelighet være et spørsmål om liv og død. Dersom den nødvendige informasjonen ikke blir gitt til den
riktige personen kan liv gå tapt. En leverandør av skytjenester lover typisk en oppetid på 99,95 \% i løpet av et år.

\subsection{Ytelse}
Løsningen må kunne skalere fra noen få brukere til flere tusen brukere. Hvis én bruker sender noen meldinger hver dag, kan det
bety opp mot en million meldinger i døgnet. Løsningen må håndtere mange meldinger uanstrengt og uten avbrytelse.

\subsection{Brukervennlighet}
Brukervennlighet i denne konteksten betyr brukervennlighet for sluttbruker, utvikler og systemadministrator.
Sluttbrukeren må føle seg trygg når løsningen benyttes, og forstå hvordan sensormålingene utføres.
For utvikler og systemadminstrator er spørsmålene hvor god dokumentasjonen er, om det er mulig å automatisere
utvikling og vedlikehold og hvordan løsningen man vedlikeholder og overvåker løsningen.

\subsection{Krav til velferdsteknologi}
%Helsedirektoratet
\subsection{Krav til prototypeplattform}

\begin{itemize}
    \item Kjøre på Node.js/JavaScript
    \item Være basert på Linux
    \item Støtte full TCP/IP-stakk
    \item Ha innebygget \gls{ble}.
\end{itemize}


%helsedir/datatilsynet
