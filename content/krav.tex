\chapter{Krav til løsning}
\label{ch:requirements}

% øke vinduet for autentisering: mange mange minutter. tilbakemelding fra sintef

\section{Kvalitetskrav}

I følge \citet{softarch} er sikkerhet, ytelse og tilgjengelighet de viktigste kvalitetskravene for skyteknologi.
Personvern er viktig for kritiske helsedata, og er noe som Datatilsynet legger stor vekt på. I tillegg til disse kvalitetskravene
er interoperabilitet og brukervennlighet trukket fram som kvalitetskrav. Basert på bakgrunnsmaterialet og case-studien,
ble kvalitetskravene rangert for en frittstående skybasert velferdsteknologiløsning.

\begin{enumerate}
    \item Sikkerhet
    \item Personvern
    \item Interoperabilitet
    \item Tilgjengelighet
    \item Ytelse
    \item Brukervennlighet
\end{enumerate}

Disse kvalitetskravene er beskrevet nøyere i de neste delkapitlene.

\subsection{Sikkerhet}
Systemet skal være beskyttet mot \gls{mitm} og tyvlytting, og skal ha tilgangskontroll på alle ressurser.
Alle kommuniserende enheter skal være autentisert for å unngå "forging of data". % @TODO fix
Det skal være intern tilgangskontroll for utviklere, helsepersonell og tredjepartsløsninger.

\subsection{Personvern}
Datatilsynet foreslår at alle nye velferdsteknologisystemer lages med innebygd personvern (\textit{privacy by design}),
noe som vil si at det tas hensyn til personvern i alle fasene av utviklingsløpet \citep{datatilsynet_privacy}. Det er mye billigere
enn å endre systemet for øke personvernshensynet i etterkant.

\subsection{Interoperabilitet}
Det er forventet at andre og tidligere utviklede systemer må integreres i løsningen. Derfor trenger løsningen flere forskjellige
grensesnitt som gjør det mulig å kommunisere med andre plattformer. Norge har bestemt at Continua-rammeverket skal brukes i nye
velferdsteknologiløsninger.

\subsection{Tilgjengelighet}
I velferdsteknologi kan tilgjengelighet være et spørsmål om liv og død. Dersom den nødvendige informasjonen ikke blir gitt til den
riktige personen kan liv gå tapt. En leverandør av skytjenester lover typisk en oppetid på 99,95 \% i løpet av et år.

\subsection{Ytelse}
Løsningen må kunne skalere fra noen få brukere til flere tusen brukere. Hvis én bruker sender noen meldinger hver dag, kan det
bety opp mot en million meldinger i døgnet. Løsningen må håndtere mange meldinger uanstrengt og uten avbrytelse.

\subsection{Brukervennlighet}
Brukervennlighet i denne konteksten betyr brukervennlighet for sluttbruker, utvikler og systemadministrator.
Sluttbrukeren må føle seg trygg når løsningen benyttes, og forstå hvordan sensormålingene utføres.
For utvikler og systemadminstrator er spørsmålene hvor god dokumentasjonen er, om det er mulig å automatisere
utvikling og vedlikehold og hvordan løsningen man vedlikeholder og overvåker løsningen.

\subsection{Krav til velferdsteknologi}
%Helsedirektoratet
\subsection{Krav til prototypeplattform}

\begin{itemize}
    \item Kjøre på Node.js/JavaScript
    \item Være basert på Linux
    \item Støtte full TCP/IP-stakk
    \item Ha innebygget \gls{ble}.
\end{itemize}


%helsedir/datatilsynet

% @TODO: https://www.datatilsynet.no/Sektor/Helse-og-omsorg/Velferdsteknologi/#
% https://www.datatilsynet.no/Regelverk/EUs-personvernforordning/hva-betyr/
% https://www.datatilsynet.no/Teknologi/Innebygd-personvern/
